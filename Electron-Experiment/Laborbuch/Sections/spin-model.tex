\section{Análisis de la Interacción de Espín}
\label{ssec:modelo_spin}

Hasta ahora, se ha modelado al electrón como una carga puntual clásica. Sin
embargo, el electrón es una partícula cuántica con espín $\frac{1}{2}$, lo
que le confiere un momento dipolar magnético intrínseco $\boldsymbol{\mu}$.
Esta propiedad podría, en principio, dar lugar a una fuerza adicional si el
campo magnético no es uniforme.

El momento magnético del electrón es $\boldsymbol{\mu} = g_s \frac{-e}{2m} \boldsymbol{S}$,
donde $\boldsymbol{S}$ es el vector de espín y $g_s \approx 2$ es el
factor g del electrón. En presencia de un campo magnético con gradiente,
esto genera la \emph{fuerza de Stern-Gerlach}:
%
\begin{equation}
	\boldsymbol{F}_{\text{espín}} = \nabla (\boldsymbol{\mu} \cdot \boldsymbol{B}^*),
	\label{eq:fuerza_spin}
\end{equation}
%
donde $\boldsymbol{B}^*$ es el vector dual al bivector de campo magnético
$\boldsymbol{B}$. A continuación se argumenta por qué esta fuerza es
despreciable en comparación con la fuerza de Lorentz.

\subsection{Comparación de Magnitudes}

En el \emph{Modelo I} (campo uniforme), $\nabla \boldsymbol{B}^* = 0$ por
definición, por lo que $\boldsymbol{F}_{\text{espín}}$ es idénticamente
nula. En los modelos más realistas con gradientes de campo (II y III), se
puede estimar la razón entre la fuerza de espín y la fuerza de Lorentz:
%
$$
\frac{|\boldsymbol{F}_{\text{espín}}|}{|\boldsymbol{F}_{\text{Lorentz}}|}
\approx \frac{|\nabla (\boldsymbol{\mu} \cdot \boldsymbol{B}^*)|}
{|e\boldsymbol{v} \times \boldsymbol{B}^*|}
\approx \frac{\mu_B |\nabla \boldsymbol{B}^*|}{e v |\boldsymbol{B}^*|}
$$
%
donde $\mu_B = e\hbar/2m$ es el magnetón de Bohr. Suponiendo que el campo
varía en una escala de longitud $L \sim R$ (el radio de las bobinas),
entonces $|\nabla \boldsymbol{B}^*| \approx |\boldsymbol{B}^*|/R$. La razón se
simplifica a:
%
$$
\frac{|\boldsymbol{F}_{\text{espín}}|}{|\boldsymbol{F}_{\text{Lorentz}}|}
\approx \frac{\hbar}{2m v R} \approx 10^{-11}
$$
%
Este cálculo, usando los parámetros del montaje, demuestra que la fuerza
de Lorentz es aproximadamente once órdenes de magnitud mayor que la
fuerza derivada del espín, justificando plenamente su omisión.

\subsection{Efecto de la Precesión de Larmor}

Independientemente de la estimación anterior, existe un segundo argumento.
En presencia de un campo $\boldsymbol{B}$, el momento magnético
$\boldsymbol{\mu}$ del electrón no permanece fijo, sino que experimenta un
torque que lo hace precesar alrededor de la dirección de $\boldsymbol{B}^*$
a la \emph{frecuencia de Larmor}, $\omega_L$.

Dado que la fuerza $\boldsymbol{F}_{\text{espín}}$ depende de la orientación
instantánea de $\boldsymbol{\mu}$, esta fuerza también oscilará rápidamente.
En el tiempo que tarda el electrón en atravesar una región con gradiente de
campo, su espín habrá completado muchas precesiones. El efecto neto de esta
fuerza oscilante sobre la trayectoria macroscópica se promedia a cero.

Por estas dos razones, es físicamente justificable y computacionalmente
necesario ignorar la interacción del espín para los propósitos de este
estudio, y continuar tratando al electrón como una carga clásica.
