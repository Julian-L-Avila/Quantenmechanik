\section{Theoretical Models}
\label{sec:TheoreticalModels}

To describe the electron beam dynamics, we develop several theoretical models.
These models, which will be compared against experimental data, are formulated
within both classical and quantum mechanical frameworks. We begin with the
classical treatment.

\subsection{Classical Models}
\label{sub:ClassicalModels}

The classical motion of an electron is governed by the Lorentz force. In the
context of this experiment, the electromagnetic field consists of the
time-varying magnetic field from the Helmholtz coils and the consequently
induced electric field. The equation of motion for an electron of mass $m$ and
charge $-e$ is expressed in Gaussian units using geometric algebra as
\begin{equation}
	m \ddot{\boldsymbol{x}} = -e \left( \boldsymbol{E}
	+ c^{-1} \langle \dot{\boldsymbol{x}} \boldsymbol{B} \rangle_{1} \right),
	\label{eq:lorentz-force}
\end{equation}
where $\boldsymbol{x}$ is the electron's position vector, $\boldsymbol{E}$ is
the electric field vector, and $\boldsymbol{B}$ is the magnetic field
bivector. The term $\langle \dot{\boldsymbol{x}} \boldsymbol{B} \rangle_{1}$
denotes the grade-1 projection of the geometric product, corresponding to the
magnetic force component.

\subsubsection{Spatially Uniform Magnetic Field}
\label{ssub:UniformMagneticField}

As a first approximation, we model the magnetic field as spatially uniform,
implying that its gradient is zero, $\nabla \boldsymbol{B} = 0$. In a
source-free region, the fields are constrained by the homogeneous Maxwell
equations. In the geometric algebra formalism, these are
\begin{gather}
	\label{eq:faraday-gauss}
	\nabla \boldsymbol{E} = -c^{-1} \partial_t \boldsymbol{B},
	\\
	\label{eq:no-monopoles}
	\langle \nabla \boldsymbol{B} \rangle_{3} = 0.
\end{gather}
\Cref{eq:faraday-gauss} combines Faraday's law of induction
($\nabla \wedge \boldsymbol{E} = -c^{-1}\partial_t \boldsymbol{B}$) and Gauss's
law for the electric field ($\nabla \cdot \boldsymbol{E} = 0$) in a
charge-free vacuum. \Cref{eq:no-monopoles}, the no-monopole condition,
permits the definition of a vector potential $\boldsymbol{A}$ such that
$\boldsymbol{B} = \nabla \wedge \boldsymbol{A} =
\langle \nabla \boldsymbol{A} \rangle_{2}$.

The relation $\boldsymbol{E} = -c^{-1}\partial_t \boldsymbol{A}$ can then be
inferred from \Cref{eq:faraday-gauss}, assuming a suitable gauge. For a
uniform field $\boldsymbol{B}$, we adopt the symmetric gauge choice
\begin{equation}
	\boldsymbol{A} = \frac{1}{2} \boldsymbol{x} \cdot \boldsymbol{B}
	= \frac{1}{2} \langle \boldsymbol{x} \boldsymbol{B} \rangle_{1},
	\label{eq:vector-potential}
\end{equation}
which correctly yields $\langle \nabla \boldsymbol{A} \rangle_{2} =
\boldsymbol{B}$. The induced electric field is therefore
\begin{equation}
	\boldsymbol{E} = -c^{-1}\partial_t \boldsymbol{A}
	= -\frac{1}{2c} \langle \boldsymbol{x} (\partial_t \boldsymbol{B}) \rangle_{1}.
\end{equation}
Substituting this expression into the Lorentz force equation,
\Cref{eq:lorentz-force}, we obtain the equation of motion for an electron in a
uniform, time-varying magnetic field:
\begin{equation}
	m \ddot{\boldsymbol{x}} = \frac{e}{c} \left\langle \frac{1}{2}
		\boldsymbol{x} (\partial_t \boldsymbol{B}) - \dot{\boldsymbol{x}}
	\boldsymbol{B} \right\rangle_{1}.
	\label{eq:eom-uniform-b}
\end{equation}

In our experiment, the transverse magnetic field is given by the bivector
$\boldsymbol{B} = B_2(t) e_{31} + B_3(t) e_{12}$.
We consider an electron beam injected along the primary axis,
such that the velocity is $\dot{\boldsymbol{x}} = \dot{x}_1 e_1$.
Substituting these forms into the equation of motion, \Cref{eq:eom-uniform-b},
yields the component equations:
\begin{align}
	m \ddot{x}_1 &= \frac{e}{2c} \left( x_3 \dot{B}_2 - x_2 \dot{B}_3 \right),
	\label{eq:eom-x1} \\
	m \ddot{x}_2 &= \frac{e}{c} \left( \frac{1}{2} x_1 \dot{B}_3 - \dot{x}_1 B_3
	\right),
	\label{eq:eom-x2} \\
		m \ddot{x}_3 &= \frac{e}{c} \left( -\frac{1}{2} x_1 \dot{B}_2 + \dot{x}_1 B_2
		\right).
	\label{eq:eom-x3}
\end{align}
This system of coupled linear ordinary differential equations describes the
electron's trajectory under the assumed conditions.
