\subsection{Modelo II: Campo a partir de la Ley de Biot-Savart}
\label{ssec:modelo_biot_savart}

En este modelo, se abandona la aproximación de campo uniforme para calcular
el campo magnético de forma más fundamental. Se utiliza la ley de
Biot-Savart para cada espira de las bobinas.

Se considera una única espira circular de radio $R$ en el plano inferior
del par de bobinas del eje $z$, ubicado en $z = -R/2$. Su geometría se
parametriza por el vector $\boldsymbol{l}_1(\theta)$:
%
\begin{equation}
	\boldsymbol{l}_1(\theta) = R \cos(\theta) e_1 + R \sin(\theta) e_2
	- \frac{R}{2} e_3.
\end{equation}
%
El campo magnético bivectorial que esta espira genera en un punto
arbitrario $\boldsymbol{r} = x_1 e_1 + x_2 e_2 + x_3 e_3$ está dado por la
ley de Biot-Savart, integrada sobre el lazo cerrado:
%
\begin{equation}
	\boldsymbol{B}_1(t, \boldsymbol{r}) = \frac{\mu_0 N I_1(t)}{4 \pi} \oint
	\frac{\mathrm{d}\boldsymbol{l}_1 \wedge (\boldsymbol{r} - \boldsymbol{l}_1)}
	{|\boldsymbol{r} - \boldsymbol{l}_1|^3},
	\label{eq:biot_savart_bivector}
\end{equation}
%
donde $N$ es el número de vueltas de la bobina y $I_1(t)$ es la corriente
que circula por ella.

El campo de la segunda bobina del par, ubicada coaxialmente en $z = +R/2$,
se deduce por simetría. Si la corriente $I_2(t)$ en la bobina superior
circula en la misma dirección que $I_1(t)$ (configuración Helmholtz),
la relación de simetría es distinta; no obstante, la forma del campo
resultante es bien conocida por su alta uniformidad en la región
central, para la cual existen soluciones analíticas estándar.

Por otro lado, si se considera una configuración anti-Helmholtz (corrientes
opuestas), el sistema presenta una simetría de inversión. En este caso, el
campo de la segunda bobina, $\boldsymbol{B}_2$, se relaciona con el de la
primera mediante una reflexión sobre el origen:
%
\begin{equation}
	\boldsymbol{B}_2(t, \boldsymbol{r}) =
	-\boldsymbol{B}_1(t, -\boldsymbol{r}, I_1(t)).
	\label{eq:simetria_inversion_B}
\end{equation}
%
Esta importante simplificación surge de la simetría axial de las bobinas
combinada con la simetría de paridad de la configuración. El campo total
del par de bobinas es entonces la superposición
$\boldsymbol{B}_{\text{par}}(t, \boldsymbol{r}) = \boldsymbol{B}_1 + \boldsymbol{B}_2$.
El mismo análisis se aplica a los otros pares de bobinas a lo largo de los
demás ejes.

\subsubsection{Campo Eléctrico en la Configuración anti-Helmholtz}
\label{sssec:campo_electrico_ah}

Aunque el análisis cuantitativo previo ya justifica la dominancia de la
fuerza magnética, la configuración anti-Helmholtz presenta una propiedad
de simetría que refuerza esta conclusión.

Debido a la simetría axial y de inversión de esta configuración, el
potencial vectorial magnético $\boldsymbol{A}$ es nulo a lo largo del eje
de las bobinas. Como consecuencia directa, el campo eléctrico inducido,
$\boldsymbol{E} = -\partial_t \boldsymbol{A}$, también es cero sobre dicho
eje. Esto implica que en la región central del montaje, la más relevante
para la trayectoria del haz, el efecto del campo eléctrico es aún más
despreciable.

\subsection{Modelo III: Aproximación de Campo Cuadrupolar}
\label{ssec:modelo_cuadrupolar}

El campo generado por un par de bobinas en configuración anti-Helmholtz
puede ser aproximado cerca de su centro por un campo de gradiente lineal,
característico de un cuadrupolo magnético. Al superponer los campos de
varios pares, se obtiene un campo cuadrupolar tridimensional.

Una aproximación lineal para dicho campo es:
%
\begin{equation}
	\boldsymbol{B}(\boldsymbol{r}, t) = b(t)
	\left( - x_1 e_{23} + \frac{1}{2} x_2 e_{31} + \frac{1}{2} x_3 e_{12} \right).
	\label{eq:quadrupole_field}
\end{equation}
%
Este campo satisface la condición de Maxwell $\langle \nabla \boldsymbol{B} \rangle_3 = 0$.
La dependencia temporal del campo se engloba en el coeficiente $b(t)$.

Este tipo de campo de gradiente es la base de las trampas magnéticas.
Cuando las corrientes son tales que el campo apunta hacia el origen, se
crea un pozo de potencial que confina el haz (efecto de \emph{enfoque}).
Inversamente, si el campo apunta hacia afuera, las partículas son
repelidas del centro (\emph{desenfoque}).

Con corrientes alternas, el sistema oscila entre un estado de enfoque y
desenfoque. Esto puede resultar en un confinamiento dinámico neto del
haz, un principio análogo al de las trampas de Paul para partículas
cargadas.
