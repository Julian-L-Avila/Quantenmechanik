\chapter{Discusión}
\label{sec:discusion}

Los resultados experimentales presentados confirman en gran medida los
modelos teóricos. Las figuras de Lissajous y los patrones de enfoque se
corresponden bien con las predicciones para campos uniformes y de
gradiente, respectivamente. Sin embargo, el resultado más revelador es el
desdoblamiento del haz en la configuración mixta, cuyo análisis merece una
discusión aparte y detallada.

\section{Análisis del Desdoblamiento del Haz: Análogo Clásico vs. Efecto Cuántico}
\label{ssec:analisis_desdoblamiento}

La observación de un haz que se desdobla en dos puntos discretos evoca
inmediatamente al célebre experimento de Stern-Gerlach (SG), un pilar de
la mecánica cuántica que demostró la cuantización del espín. A pesar de
esta semejanza visual, una análisis riguroso revela que el fenómeno
observado en este montaje es de naturaleza puramente clásica. A
continuación, se exponen las características fundamentales que impiden
catalogarlo como un efecto SG.

\subsection{La Naturaleza de la Fuerza Dominante}

El argumento más contundente reside en la naturaleza de la partícula y la
fuerza que gobierna su trayectoria. El experimento SG original utiliza
partículas \emph{neutras} (átomos de plata) para aislar la interacción
entre el momento dipolar magnético y el gradiente del campo. La fuerza
responsable de la separación es la fuerza de Stern-Gerlach,
$\boldsymbol{F}_{\text{espín}} = \nabla (\boldsymbol{\mu} \cdot \boldsymbol{B}^*)$.

En nuestro experimento, la partícula es un electrón, una partícula
\emph{cargada}. Como se demostró cuantitativamente en la
\cref{ssec:modelo_spin}, la fuerza de Lorentz que actúa sobre la carga
del electrón es aproximadamente $10^{11}$ veces mayor que cualquier
posible fuerza sobre su espín. Por lo tanto, la dinámica observada es,
casi en su totalidad, el resultado de la fuerza de Lorentz. El desdoblamiento
no proviene de una interacción con el espín, sino de una compleja
interacción de la \emph{carga} del electrón con los campos externos.

\subsection{Campos Dinámicos (AC) vs. Estáticos (DC)}

El experimento SG canónico requiere un campo magnético \emph{estático}
(DC) con un gradiente espacial muy pronunciado y constante en el tiempo.
La separación que produce es, por tanto, estática: los átomos son
desviados permanentemente hacia una de dos trayectorias.

Nuestro montaje, en cambio, se basa fundamentalmente en campos
\emph{dinámicos} (AC) que oscilan a una frecuencia de \qty{50}{\hertz}.
El patrón observado no es una separación estática, sino una \emph{oscilación}
del haz entre dos puntos. El electrón no elige una de dos trayectorias
fijas, sino que es guiado alternativamente a una y otra posición por los
campos que varían en el tiempo. Esta dependencia temporal es ajena al
concepto del experimento SG.

\subsection{Separación Clásica vs. Cuantizada}

La separación en el experimento SG es una manifestación directa de la
\emph{cuantización del espacio}. Para una partícula de espín-1/2, existen
únicamente dos proyecciones de espín posibles, lo que resulta en
exactamente dos haces. La magnitud de esta separación depende de
constantes fundamentales de la naturaleza.

En nuestro experimento, la posición de los dos puntos observados no está
fijada por constantes fundamentales, sino que depende de los parámetros
\emph{clásicos} y continuamente ajustables del sistema: la geometría de
las bobinas, la amplitud y frecuencia de las corrientes, y la energía
inicial del haz. Se postula que la topología del campo mixto crea un
\emph{potencial efectivo} con dos mínimos locales. La posición de estos
mínimos puede ser modificada alterando los voltajes y corrientes, algo
que sería imposible en un desdoblamiento cuántico verdadero.

En conclusión, el fenómeno observado es un fascinante \emph{análogo clásico}
del experimento de Stern-Gerlach. Demuestra cómo una compleja ingeniería
de campos electromagnéticos clásicos y dependientes del tiempo puede
generar un comportamiento que simula visualmente un resultado cuántico. Lejos
de ser una medición del espín, es un testimonio de la riqueza de la dinámica
no lineal en el electromagnetismo clásico.
