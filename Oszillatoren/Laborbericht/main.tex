\documentclass[pdflatex,sn-mathphys-num]{sn-jnl}

\usepackage[utf8]{inputenc}
\usepackage[T1]{fontenc}
\usepackage[spanish]{babel}
\usepackage{graphicx}
\usepackage{multirow}
\usepackage{amsmath,amssymb,amsfonts}
\usepackage{amsthm}
\usepackage{mathrsfs}
\usepackage{xcolor}
\usepackage{textcomp}
\usepackage{booktabs}
\usepackage{algorithm}
\usepackage{algorithmicx}
\usepackage{algpseudocode}
\usepackage{microtype}
\usepackage{csquotes}
\usepackage{hyperref}
\usepackage{pgfplotstable}
\usepackage[spanish]{cleveref}
\usepackage{siunitx}
\usepackage{subfig}
\usepackage{bm}
\usepackage{multirow}

\pgfplotsset{compat=1.18}
\pgfkeys{/pgf/number format/1000 sep={\,}}

\AtBeginDocument{\decimalpoint}
\renewcommand{\arraystretch}{1.2}
\sisetup{group-digits=true,
  group-separator={\,},
  separate-uncertainty}

\hypersetup{
  colorlinks=true,
  linkcolor=black,
  urlcolor=blue,
  pdftitle={Report},
  pdfauthor={Rodriguez S, Torres L, Avila J},
  pdfsubject={Coupled Oscillators},
}
\urlstyle{same}

\pgfplotstableread[col sep=comma, header=true]{
  i, y_cm_i, m_i
  1, 14.2, 600.8
  2, 28.0, 1216.3
  3, 14.0, 601.8
}\mydata

\pgfplotstableread[col sep=comma, header=true]{
  config, mode, freq1, freq2, freq3
  \multirow{3}{*}{5-1}, 100, 1.2715, 0.8408, 0.0205
  , 010, 0.8308, 0.8308, 0.8308
  , 101, 1.2771, 0.8375, 1.1934
  \multirow{3}{*}{6-1}, 100, 1.3147, 0.8383, 1.2003
  , 010, 0.8415, 0.8415, 0.8415
  , 101, 1.3144, 0.8348, 1.1901
  \multirow{3}{*}{6-2}, 100, 1.3303, 0.8455, 1.2288
  , 010, 0.8378, 0.8378, 0.8378
  , 101, 1.3189, 0.8402, 1.2114
  \multirow{3}{*}{6-4}, 100, 1.3263, 0.8547, 1.3263
  , 010, 0.8498, 0.8498, 0.8498
  , 101, 1.3250, 0.8505, 1.2713
  \multirow{3}{*}{6-6}, 100, 1.3144, 0.8552, 1.3303
  , 010, 0.8519, 0.8519, 0.8519
  , 101, 1.3147, 0.8574, 1.3474
}\datafreq

\pgfplotstableread[col sep=comma, header=true]{
  config, f1, f2, f3
  5-1, 1.1669, 0.8305, 1.2647
  6-1, 1.1669, 0.8343, 1.3120
  6-2, 1.1815, 0.8366, 1.3120
  6-4, 1.2394, 0.8437, 1.3122
  6-6, 1.3100, 0.8513, 1.3356
}\theoric

\pgfplotstableread[col sep=comma, header=true]{
  config, fe, ft, err
  \multirow{3}{*}{5-1}, 0.8305, 0.8305, 0.0361
  , 1.1934, 1.1669, 2.2205
  , 1.2771, 1.2647, 0.9709
  \multirow{3}{*}{6-1}, 0.8415, 0.8343, 0.8556
  , 1.1901, 1.1669, 1.9494
  , 1.3144, 1.3120, 0.1826
  \multirow{3}{*}{6-2}, 0.8378, 0.8366, 0.1432
  , 1.2114, 1.1815, 2.4682
  , 1.3189, 1.3120, 0.5232
  \multirow{3}{*}{6-4}, 0.8498, 0.8437, 0.7178
  , 1.2713, 1.2394, 2.5092
  , 1.3250, 1.3122, 0.9660
  \multirow{3}{*}{6-6}, 0.8519, 0.8513, 0.0704
  , 1.3147, 1.3100, 0.3575
  , 1.3474, 1.3356, 0.8758
}\comparison

\title[Caracterización Experimental de las Frecuencias de Modos Normales en Péndulos Acoplados Usando la Aproximación de Pequeñas Oscilaciones]{Caracterización Experimental de las Frecuencias de Modos Normales en Péndulos Acoplados Usando la Aproximación de Pequeñas Oscilaciones}
\author[1]{\fnm{Sebastián} \sur{Rodríguez}}
\author[1]{\fnm{Laura} \sur{Torres}}
\author[1]{\fnm{Julian} \sur{Avila}}
\affil[1]{\orgdiv{Física}, \orgname{Universidad Distrital Francisco José de Caldas}}

\raggedbottom

\begin{document}

\abstract{
  Se investigó experimentalmente el comportamiento dinámico y
  los modos normales de oscilación de un sistema de tres péndulos
  acoplados, aplicando la aproximación de pequeñas oscilaciones.
  La metodología incluyó el diseño y construcción del montaje, la
  caracterización de componentes, y la adquisición de datos de
  desplazamiento angular en cinco configuraciones y tres condiciones
  iniciales.
  Las ecuaciones de movimiento se derivaron mediante sumatoria de torques
  y la aproximación de ángulos pequeños.
  El análisis de datos se realizó con Transformada Rápida de Fourier (FFT)
  para identificar frecuencias principales, comparándolas con valores
  teóricos obtenidos del sistema linealizado.
  Los resultados experimentales identificaron las frecuencias de oscilación
  para cada péndulo, demostrando su dependencia de la configuración y
  condiciones iniciales.
  El péndulo central mostró una frecuencia consistente (\qty{0.844}{\Hz}),
  mientras que los laterales exhibieron dualidad de frecuencias dominantes.
  La comparación cuantitativa con la teoría reveló una precisión notable
  (errores entre \qty{0.036}{\percent} y \qty{2.509}{\percent}).
  Se confirmó el papel crítico de la geometría de acoplamiento en la
  determinación de frecuencias y transmisión de energía, observándose
  pulsaciones.
  Se identificó una frecuencia anómala baja (\qty{0.0021}{\Hz})
  no predicha por el modelo lineal, sugiriendo efectos no lineales o
  amortiguamiento.
  Se valida la aproximación de pequeñas oscilaciones y la identificación
  experimental de modos normales.
}

\maketitle

\section{Introducción}

En el estudio de la mec\'anica cl\'asica, la comprensi\'on del
comportamiento din\'amico de los sistemas es fundamental.
Particularmente, los sistemas oscilatorios juegan un papel
crucial en una amplia gama de fen\'omenos, desde el movimiento
de un p\'endulo simple hasta las vibraciones de estructuras
complejas.
A menudo, estos sistemas involucran m\'ultiples componentes que interact\'uan
entre s\'i, dando lugar a un comportamiento acoplado que es m\'as complejo que
la suma de sus partes individuales.

Una herramienta anal\'itica extraordinariamente potente para
desentra\~nar esta complejidad es la aproximaci\'on
de peque\~nas oscilaciones.
Esta aproximaci\'on establece que cualquier sistema f\'isico, bajo ciertas
condiciones y cerca de una posici\'on de equilibrio estable, puede ser
linealizado.
Dicho de otro modo, el sistema se comporta de manera an\'aloga a un conjunto de
osciladores arm\'onicos simples acoplados.
La fuerza de esta aproximaci\'on reside en que permite transformar problemas
intrincados en un marco matem\'atico m\'as manejable, revelando las frecuencias
naturales y los modos normales de oscilaci\'on del sistema.

Los modos normales representan patrones de movimiento colectivo en los que todas
las partes del sistema oscilan con la misma frecuencia y fase relativa constante.
Visualizar y comprender estos modos es esencial, ya que ofrecen una perspectiva
fundamental sobre la estabilidad y la respuesta din\'amica de un sistema.
En mec\'anica cl\'asica, los modos normales son la clave para entender fen\'omenos
como la resonancia, la propagaci\'on de ondas en medios continuos y la
transferencia de energ\'ia entre osciladores \cite{Goldstein}.

Este concepto trasciende la f\'isica cl\'asica y sienta las bases para la
comprensi\'on de fen\'omenos cu\'anticos.
La idea de estados propios discretos y sus correspondientes valores propios en
sistemas cu\'anticos tiene un paralelismo conceptual directo con los modos
normales y las frecuencias naturales en la mec\'anica cl\'asica.
Es, por tanto, un primer paso indispensable para adentrarse en la mec\'anica
cu\'antica y la comprensi\'on de sistemas de m\'ultiples part\'iculas
\cite{Griffiths2017}.

En este trabajo, se presenta un sistema de tres p\'endulos acoplados dise\~nado
espec\'ificamente para visualizar y estudiar experimentalmente sus modos normales
de oscilaci\'on.
A trav\'es de este sistema, se busca proporcionar una plataforma que permita
explorar directamente los principios de la aproximaci\'on de peque\~nas
oscilaciones, la determinaci\'on de frecuencias naturales y la identificaci\'on
de los modos normales.

\section{Ecuaciones de Movimiento}

Para describir la din\'amica del sistema compuesto por tres p\'endulos
f\'isicos, se establecen consideraciones iniciales que simplifican el
an\'alisis y la deducci\'on de las ecuaciones de movimiento.
Inicialmente, los p\'endulos se representan en sus puntos de
equilibrio, donde los ejes de rotaci\'on est\'an separados por una
distancia $a$, correspondiente a la longitud natural de los muelles.
Este esquema se ilustra en la \cref{fig:esquema_equilibrio}.

\begin{figure}[htbp!]
  \centering
  \includegraphics[width=0.8\linewidth]{./Figures/system-diagram.pdf}
  \caption{Esquema de los p\'endulos en equilibrio. La distancia
  entre los ejes de rotaci\'on es $a$.}
  \label{fig:esquema_equilibrio}
\end{figure}

Las ecuaciones de movimiento del sistema se obtienen mediante la
aplicaci\'on de la sumatoria de torques, considerando tanto las
contribuciones de los resortes como la de la fuerza gravitacional
. Si bien los formalismos Lagrangiano y Hamiltoniano
ofrecen un enfoque m\'as general y elegante para sistemas complejos,
para el presente sistema de p\'endulos, la aplicaci\'on directa de
la segunda ley de Newton para rotaci\'on (sumatoria de torques)
resulta en un m\'etodo igualmente riguroso y conceptualmente claro
para la derivaci\'on de las ecuaciones de movimiento.
Es necesario, entonces, encontrar una expresi\'on para el torque
gravitacional en funci\'on de cualquier \'angulo $\theta_i$, as\'i
como una expresi\'on para las distancias del tipo $x_0 + \Delta x$,
que representan la elongaci\'on de los resortes en funci\'on de los
\'angulos $\theta_i$ y $\theta_{i\pm 1}$. Las barras se
enumeran de izquierda a derecha como 1, 2 y 3, seg\'un se muestra en la
\cref{fig:enumeracion_barras}.

\begin{figure}[htbp!]
  \centering
  \includegraphics[width=0.8\linewidth]{Figures/Ilustración_sin_título 4.pdf}
  \caption{Numeraci\'on de las barras del sistema de p\'endulos:
  1 (izquierda), 2 (central), 3 (derecha).}
  \label{fig:enumeracion_barras}
\end{figure}

Al desplazar las barras 1 y 2 de su posici\'on de equilibrio, se
observa que la tangente de cada barra se desv\'ia un \'angulo
$\theta_i$ del eje horizontal. Esta condici\'on permite
construir un tri\'angulo rect\'angulo con las componentes $x_i$ e
$y_i$ de la posici\'on del punto de aplicaci\'on de la fuerza
el\'astica para cada barra, generalizada a cualquier fracci\'on de
la longitud $l$, tal como se ilustra en la \cref{fig:triangulo_posicion}
.

\begin{figure}[htbp!]
  \centering
  \includegraphics[width=0.8\linewidth]{Figures/Ilustración_sin_título 5.pdf}
  \caption{Construcci\'on geom\'etrica para determinar la posici\'on
  del punto de aplicaci\'on de la fuerza el\'astica en una barra.}
  \label{fig:triangulo_posicion}
\end{figure}

Las coordenadas $x_k$ e $y_k$ para $k = 1, 2$ est\'an definidas como:
\begin{align}
  x_k &= c l \sin(\theta_{k}) \label{eq:xk} \\
  y_k &= c l \cos(\theta_{k}) \label{eq:yk}
\end{align}
donde $c$ y $d$ son fracciones de la longitud de la barra $l$ que
indican la posici\'on de los puntos de acople de los resortes.
El tri\'angulo resultante est\'a definido por un \'angulo $\psi_1$,
el cual cumple la relaci\'on $\beta_1 + \psi_1 = \theta_1$, donde
$\beta_1$ corresponde al \'angulo entre la tangente de la barra 1 y
la l\'inea recta que conecta los puntos de aplicaci\'on.
A partir de este tri\'angulo, se obtiene la expresi\'on para la
elongaci\'on del resorte entre las barras 1 y 2:
\begin{equation}
  x_{01} + \Delta x_{1} = \sqrt{(y_2 - y_1)^2 + (a + x_2 - x_1)^2}
  \label{eq:elongacion1}
\end{equation}
con $x_{01}$ como la longitud natural del resorte.
Y el \'angulo $\psi_1$ est\'a dado por:
\begin{equation}
  \psi_1 = \arctan\left( \frac{y_2 - y_1}{a - x_1 + x_2} \right)
  \label{eq:psi1}
\end{equation}
De forma an\'aloga, para el resorte entre las barras 2 y 3, se
utiliza un punto de aplicaci\'on $d l$, resultando en:
\begin{align}
  \psi_2 &= \arctan \left( \frac{y_4 - y_3}{a - x_3 + x_4} \right) \label{eq:psi2}, \\
  x_{02} + \Delta x_{2} &= \sqrt{(y_4 - y_3)^2 + (a + x_4 - x_3)^2} \label{eq:elongacion2}
\end{align}
Donde para $j = 3, 4$, las coordenadas $x_j$ e $y_j$ son:
\begin{align}
  x_j &= d l \sin(\theta_{j}) \label{eq:xj} \\
  y_j &= d l \cos(\theta_{j}) \label{eq:yj}
\end{align}
El torque gravitacional sobre cada barra $i$ est\'a dado por:
\begin{equation}
  \tau_{g_i} = y_{\text{cm}_i} \, m_i \, g \sin(\theta_i) \label{eq:tau_g}
\end{equation}
Donde $y_{\text{cm}_i}$ es la posici\'on del centro de masa de la
barra $i$. Para los torques el\'asticos, se deben
considerar los \'angulos $\beta_i$, definidos como:
\begin{align}
  \beta_1 + \psi_1 &= \theta_1, & \beta_2 + \psi_1 &= \theta_2, \label{eq:beta1_psi1}\\
  \beta_3 + \psi_2 &= \theta_2, & \beta_4 + \psi_2 &= \theta_3 \label{eq:beta3_psi2}
\end{align}
Es importante notar que, debido a las distintas posiciones de
aplicaci\'on de los resortes entre los pares de barras, aparecen
cuatro \'angulos $\beta_i$, donde los \'indices 2 y 3 se refieren
a la barra central. La sumatoria de torques para cada
p\'endulo, considerando el sentido antihorario como positivo, resulta
en las siguientes ecuaciones de movimiento:
\begin{align}
  \sum \tau_{m_1} &= c l k_1 (\Delta x_1) \cos(\beta_1) - m_1 g y_{\text{cm}_1} \sin(\theta_1) = I_1 \, \ddot{\theta}_1 \label{eq:tau_m1}, \\
  \sum \tau_{m_2} &= d l k_2 (\Delta x_2) \cos(\beta_3) - c l k_1 (\Delta x_1) \cos(\beta_2) - m_2 g y_{\text{cm}_2} \sin(\theta_2) = I_2 \, \ddot{\theta}_2 \label{eq:tau_m2}, \\
  \sum \tau_{m_3} &= -d l k_2 (\Delta x_2) \cos(\beta_4) + m_3 g y_{\text{cm}_3} \sin(\theta_3) = I_3 \, \ddot{\theta}_3 \label{eq:tau_m3}
\end{align}

\subsection*{Aproximaciones para \'Angulos Peque\~nos}

Con el fin de obtener una soluci\'on anal\'itica aproximada que permita
calcular las frecuencias naturales del sistema, se aplican
las siguientes aproximaciones para \'angulos peque\~nos ($\theta \ll 1$ rad):
\begin{align*}
  \sin(\theta) &\approx \theta, \\
  \cos(\theta) &\approx 1, \\
  \tan(\theta) &\approx \theta.
\end{align*}
Aplicando estas aproximaciones, las coordenadas $x_k$ e $y_k$ se
simplifican a:
\begin{align*}
  x_1 &= c l \theta_1, & x_2 &= c l \theta_2 \\
  x_3 &= d l \theta_2, & x_4 &= d l \theta_3 \\
  y_1 &= c l, & y_2 &= c l \\
  y_3 &= d l, & y_4 &= d l
\end{align*}
Si se cumple la condici\'on de que las longitudes naturales de los
resortes son iguales a la separaci\'on entre los ejes, es decir,
$x_{01} = x_{02} = a$, las elongaciones de los resortes $\Delta x_1$
y $\Delta x_2$ se simplifican a:
\begin{align}
  \Delta x_1 &= cl(\theta_2 - \theta_1) \label{eq:deltax1_approx}, \\
  \Delta x_2 &= dl(\theta_3 - \theta_2) \label{eq:deltax2_approx}.
\end{align}
Adem\'as, si se asume que $\arctan(\psi_i) \approx \psi_i \approx 0$,
lo que implica que la fuerza el\'astica act\'ua aproximadamente en
la direcci\'on horizontal, entonces los \'angulos $\beta_i$ se
aproximan a los \'angulos $\theta_i$ ($\beta_i \approx \theta_i$).
Esto simplifica la sumatoria de torques a:
\begin{align}
  I_1 \ddot{\theta}_1 &= (cl)^2 k_1 (\theta_2 - \theta_1) - m_1 g y_{\text{cm}_1} \theta_1 \label{eq:eq_mov_lin1}, \\
  I_2 \ddot{\theta}_2 &= (dl)^2 k_2 (\theta_3 - \theta_2) - (cl)^2 k_1 (\theta_2 - \theta_1) - m_2 g y_{\text{cm}_2} \theta_2 \label{eq:eq_mov_lin2}, \\
  I_3 \ddot{\theta}_3 &= -(dl)^2 k_2 (\theta_3 - \theta_2) + m_3 g y_{\text{cm}_3} \theta_3 \label{eq:eq_mov_lin3}.
\end{align}
Finalmente, el sistema de ecuaciones diferenciales lineales queda
expresado como:
\begin{align}
  \ddot{\theta}_1 &=\; \theta_1 \left( \frac{(cl)^2 k_{1} - y_{\text{cm}_1} m_1 g}{I_1} \right) + \theta_2 \left( -\frac{k_1 (cl)^2}{I_1} \right) \label{eq:eom1} \\
  \ddot{\theta}_2 &=\; \theta_1 \left( \frac{k_1 (cl)^2}{I_2} \right) + \theta_2 \left( \frac{k_2 (dl)^2 - k_1 (cl)^2 + y_{\text{cm}_2} m_2 g}{I_2} \right) + \theta_3 \left( -\frac{k_2 (dl)^2}{I_2} \right) \label{eq:eom2} \\
  \ddot{\theta}_3 &=\; \theta_2 \left( -\frac{k_2 (dl)^2}{I_3} \right) + \theta_3 \left( \frac{k_2 (dl)^2 - y_{\text{cm}_3} m_3 g}{I_3} \right) \label{eq:eom3}
\end{align}
Estas ecuaciones forman un sistema lineal de ecuaciones
diferenciales ordinarias acopladas, que pueden ser presentadas
como una ecuaci\'on matricial \cref{eq:matrix-form} donde
$\bm{M}$, $\bm{K}$ y $\bm{\Theta}$ representan una matriz cuadrada
diagonal con los t\'erminos de inercia, una matriz cuadrada con los
t\'erminos de acople, y un vector columna con las coordenadas
angulares, respectivamente.
\begin{equation}
  \mathbf{M} \ddot{\bm{\Theta}} + \mathbf{K} \bm{\Theta} = \mathbf{0}
  \label{eq:matrix-form}
\end{equation}
Este sistema linealizado permite determinar las frecuencias
naturales del sistema al resolver el problema de valores propios
asociado.

\subsection*{Modos normales}

A partir de la ecuaci\'on en forma matricial \cref{eq:matrix-form} se
lleva a un problema de valores propios para determinar los modos
normales. Esto se logra con la posibilidad de invertir la matriz con
los t\'erminos inerciales e identificar la segunda derivada como el
operador:
\begin{equation}
  \mathbf{D}^2_t \bm{\Theta} = - \mathbf{\Omega} \bm{\Theta}
  \label{eq:eigenproblem}
\end{equation}
donde $\mathbf{D}^2_t$ representa el operador de la segunda derivada
temporal y $\mathbf{\Omega} = \mathbf{M}^{-1}\mathbf{K}$ es la matriz
din\'amica del sistema. Dada la complejidad del c\'alculo anal\'itico
para un sistema de tres grados de libertad, se emplea una soluci\'on
num\'erica usando la librer\'ia de Python SciPy para determinar
las frecuencias de los modos normales y sus correspondientes vectores
propios.

\section{Metodolog\'ia Experimental}

La metodolog\'ia experimental se estructur\'o en varias etapas
fundamentales: dise\~no y construcci\'on del montaje,
caracterizaci\'on de sus componentes y adquisici\'on de datos
mediante sensores angulares.

\subsection{Construcci\'on del Montaje}

Para la construcci\'on del sistema se utilizaron tres barras
met\'alicas de alta densidad, lo que contribuye a la rigidez del
sistema y a una mejor definici\'on de su centro de masa.
Las barras laterales tienen una longitud de
$l/2 = \qty{28.0(1)}{\centi\metre}$, mientras que la barra central
posee una longitud de $l = \qty{56.0(1)}{\centi\metre}$.
Las masas y las posiciones de los centros de masa de cada barra
fueron determinadas experimentalmente y se resumen en la
\cref{tab:bars-dat}. La posici\'on del centro de masa,
$y_{\text{cm},i}$, se mide desde el punto de pivote de cada p\'endulo.

\begin{table}[htbp!]
  \caption{Parámetros físicos de las barras empleadas en el montaje.La incertidumbre para la posición del centro de masa (\(y_{\text{cm}_i}\)) es de \qty{0.1}{\centi\metre} y para la masa (\(m_i\)) es de \qty{0.1}{\gram}.}
  \centering
  \pgfplotstabletypeset[
  col sep=comma,
  zerofill,
  columns/i/.style={
    string type,
    column type={c},
    column name={\(i\)},
  },
  columns/y_cm_i/.style={
    column name={\(y_{\text{cm}_i} [\si{\centi\metre}]\)},
    precision=1,
    fixed,
    fixed zerofill,
  },
  columns/m_i/.style={
    column type={c},
    column name={\(m_i [\si{\gram}]\)},
    dec sep align,
    precision=1,
    fixed,
    fixed zerofill,
  },
  every head row/.style={
    before row=\toprule,
    after row=\midrule,
  },
  every last row/.style={
    after row=\bottomrule,
  }
  ]\mydata
  \label{tab:bars-dat}
\end{table}

Las barras fueron perforadas en m\'ultiples puntos para permitir
diversas configuraciones de acople de los resortes. La
\cref{fig:barras} muestra las barras met\'alicas preparadas para el
montaje.

\begin{figure}[htbp!]
  \centering
  \includegraphics[width=0.6\linewidth]{Figures/metal-bars.jpeg}
  \caption{Barras met\'alicas preparadas para el montaje de los
  p\'endulos, mostrando las perforaciones para el acople de resortes.}
  \label{fig:barras}
\end{figure}

\subsection{Medici\'on de la Constante El\'astica}

La constante el\'astica de los resortes empleados ($k_1$ y $k_2$) se
determin\'o experimentalmente. Se aplicaron masas conocidas a cada
resorte y se registraron los desplazamientos resultantes.
Mediante una regresi\'on lineal de los datos de fuerza
(peso aplicado) versus elongaci\'on, se obtuvieron los siguientes
valores para las constantes el\'asticas:
\begin{itemize}
  \item $k_1 = \qty{3.04(4)}{\N\per\m}$
  \item $k_2 = \qty{3.32(6)}{\N\per\m}$
\end{itemize}
El proceso gr\'afico de determinaci\'on de estas constantes se
ilustra en la \cref{fig:regresion}.

\begin{figure}[htbp!]
  \centering
  \includegraphics[width=0.75\linewidth]{Figures/springs-plot.pdf}
  \caption{Determinaci\'on de la constante el\'astica de los resortes
    mediante regresi\'on lineal de los datos de fuerza aplicada en
  funci\'on del desplazamiento.}
  \label{fig:regresion}
\end{figure}

\subsection{Integraci\'on del Sistema de Medici\'on}

Para registrar los desplazamientos angulares $\theta_i(t)$ de
cada p\'endulo, se integraron sensores angulares rotacionales
(Cassy) en cada uno de los puntos de pivote. Las barras de los
p\'endulos se fijaron a las poleas de los sensores utilizando
alambre dulce. Aunque esta metodolog\'ia de fijaci\'on podr\'ia
introducir un m\'inimo juego mec\'anico, se realiz\'o con cautela
para minimizar cualquier holgura y asegurar mediciones angulares
precisas. La \cref{fig:montaje} presenta el montaje experimental
completo con los sensores integrados.

\begin{figure}[htbp!]
  \centering
  \includegraphics[width=0.75\textwidth]{Figures/set-up.jpeg}
  \caption{Montaje experimental completo del sistema de tres
    p\'endulos acoplados, con los sensores angulares Cassy
  integrados en los pivotes.}
  \label{fig:montaje}
\end{figure}

\subsection{Toma de Datos y Configuraciones Experimentales}

Se realizaron mediciones bajo cinco configuraciones distintas de
acoplamiento de resortes, esquematizadas en la \cref{fig:configs}.
Para cada configuraci\'on, se investigaron tres tipos de
condiciones iniciales para excitar el sistema:
\begin{itemize}
  \item (001): Desplazamiento inicial \'unicamente del p\'endulo 3.
  \item (010): Desplazamiento inicial \'unicamente del p\'endulo 2.
  \item (101): Desplazamiento inicial sim\'etrico de los p\'endulos
    1 y 3 (en la misma direcci\'on y amplitud).
\end{itemize}
Cada medici\'on se registr\'o durante un intervalo aproximado de
\qty{30}{\second}, permitiendo la captura de m\'ultiples
oscilaciones completas. Los datos de los \'angulos en funci\'on
del tiempo fueron almacenados digitalmente para su posterior
an\'alisis y comparaci\'on con los resultados te\'oricos derivados
de los modos normales de oscilaci\'on.

\begin{figure}[htbp!]
  \centering
  \begin{subfigure}[b]{0.3\textwidth}
    \centering
    \includegraphics[width=\linewidth]{./Figures/15.pdf}
    \caption{Configuraci\'on 1-5}
    \label{fig:conf-1-5}
  \end{subfigure}
  \hfill
  \begin{subfigure}[b]{0.3\textwidth}
    \centering
    \includegraphics[width=\linewidth]{./Figures/16.pdf}
    \caption{Configuraci\'on 1-6}
    \label{fig:conf-1-6}
  \end{subfigure}
  \hfill
  \begin{subfigure}[b]{0.3\textwidth}
    \centering
    \includegraphics[width=\linewidth]{./Figures/26.pdf}
    \caption{Configuraci\'on 2-6}
    \label{fig:conf-2-6}
  \end{subfigure}

  \vspace{0.5cm} % Espacio vertical entre filas de subfiguras

  \begin{subfigure}[b]{0.45\textwidth} % Ajustado para centrar mejor dos figuras
    \centering
    \includegraphics[width=0.6\linewidth]{./Figures/46.pdf} % Ajustado para que no sea tan grande
    \caption{Configuraci\'on 4-6}
    \label{fig:conf-4-6}
  \end{subfigure}
  \hfill % Para espaciar igualmente si hubiera otra
  \begin{subfigure}[b]{0.45\textwidth} % Ajustado para centrar mejor dos figuras
    \centering
    \includegraphics[width=0.6\linewidth]{./Figures/66.pdf} % Ajustado para que no sea tan grande
    \caption{Configuraci\'on 6-6}
    \label{fig:conf-6-6}
  \end{subfigure}

  \caption{Representaci\'on esquem\'atica de las cinco configuraciones de
    acoplamiento de resortes estudiadas. La nomenclatura 'X-Y' en
    cada subfigura (e.g., 1-5) indica los orificios espec\'ificos
    (numerados) en los p\'endulos adyacentes donde se anclaron
  los extremos de los resortes.}
  \label{fig:configs}
\end{figure}

\section{Resultados y An\'alisis}

Tras el montaje del sistema de p\'endulos acoplados y el registro
de las mediciones mediante el sensor Cassy, se recopil\'o un
conjunto de 15 series de datos. Estas series corresponden a tres
condiciones iniciales distintas para cada una de las cinco
configuraciones experimentales estudiadas. Los datos temporales de
los \'angulos de cada p\'endulo fueron procesados utilizando un
c\'odigo en Python, con el fin de generar gr\'aficas de la evoluci\'on
angular $\theta_i(t)$ y, fundamentalmente, para determinar las
frecuencias angulares principales de oscilaci\'on mediante la
aplicaci\'on de la Transformada R\'apida de Fourier (FFT).

En la \cref{tab:frequencies} se presenta un resumen de las
frecuencias angulares principales identificadas para cada p\'endulo,
en funci\'on de la configuraci\'on del sistema y de la condici\'on
inicial aplicada. Un an\'alisis preliminar de estos valores revela
patrones interesantes: la frecuencia angular usual del p\'endulo 2
(el m\'as largo) se sit\'ua consistentemente alrededor de
\qty{0.844}{\Hz}, con una desviaci\'on est\'andar
reducida de \num{0.009}, lo que indica una notable regularidad en su
comportamiento oscilatorio a esta frecuencia.

\begin{table*}[htbp!]
	\centering
	\caption{Frecuencias angulares principales de oscilaci\'on
		($f_i$) identificadas para cada p\'endulo, seg\'un la
		configuraci\'on experimental y las condiciones iniciales aplicadas.
	}
	\label{tab:frequencies}
	\pgfplotstabletypeset[
	every head row/.style={
		before row=\toprule,
		after row=\midrule
	},
	every last row/.style={after row=\bottomrule},
	columns/config/.style={
		string type,
		column name={Configuración},
	},
	columns/mode/.style={
		string type,
		column name={Condición Inicial},
	},
	columns/freq1/.style={
		column name=$f_1 [\si{\Hz}]$,
		fixed,
		fixed zerofill,
		precision=3,
	},
	columns/freq2/.style={
		column name=$f_2 [\si{\Hz}]$,
		fixed,
		fixed zerofill,
		precision=3,
	},
	columns/freq3/.style={
		column name=$f_3 [\si{\Hz}]$,
		fixed,
		fixed zerofill,
		precision=3,
	},
	every nth row={3}{before row=\midrule},
	columns={config, mode, freq1, freq2, freq3}
	]{\datafreq}
\end{table*}

En contraste, para el p\'endulo 1, se identifican dos agrupaciones
principales de frecuencias: una en torno a \qty{1.311}{\Hz}
(desviaci\'on est\'andar de \num{0.019}) y otra cercana a
\qty{0.843}{\Hz} (desviaci\'on est\'andar de \num{0.008}). Un
comportamiento similar, aunque con valores ligeramente distintos, se
observa en el p\'endulo 3, el cual exhibe frecuencias predominantes
alrededor de \qty{1.255}{\Hz} (desviaci\'on est\'andar de \num{0.063})
y \qty{0.843}{\Hz} (desviaci\'on est\'andar de \num{0.008}).
Adicionalmente, para el p\'endulo 3, se detect\'o una frecuencia
excepcionalmente baja de \qty{0.0021}{\Hz}, la cual se present\'o
de manera aislada \'unicamente en la Configuraci\'on 5-1 bajo las
condiciones iniciales (100). La aparici\'on de m\'ultiples
frecuencias dominantes para los p\'endulos laterales (1 y 3)
sugiere la excitaci\'on selectiva de diferentes modos normales del
sistema, cuya manifestaci\'on depende de la configuraci\'on
espec\'ifica de acoplamiento y de las condiciones iniciales
impuestas.

Un an\'alisis m\'as detallado del \cref{tab:frequencies} evidencia
que para la condici\'on inicial (010) (excitaci\'on \'unica del
p\'endulo central), los tres p\'endulos tienden a oscilar con una
frecuencia dominante com\'un o muy similar, independientemente de la
configuraci\'on de acoplamiento. Esto podr\'ia indicar la
excitaci\'on preferente de un modo normal en el que los tres
p\'endulos participan de manera sincronizada. En las dem\'as
condiciones iniciales, el p\'endulo 3 generalmente muestra una
frecuencia principal inferior a la del p\'endulo 1. No obstante,
se observa una excepci\'on en la configuraci\'on 6-6, donde la
frecuencia principal del p\'endulo 3 supera a la del p\'endulo 1.

La menor frecuencia caracter\'istica del p\'endulo 2 se atribuye
principalmente a su mayor longitud y masa (y por ende, mayor momento
de inercia) en comparaci\'on con los p\'endulos laterales. Las
diferencias observadas en las frecuencias principales entre los
p\'endulos 1 y 3, as\'i como su variaci\'on entre las distintas
configuraciones, son consecuencia de c\'omo los diferentes puntos de
acople de los resortes modifican la interacci\'on entre ellos. La
altura a la que se fijan los resortes en las barras de los
p\'endulos influye en la magnitud de los torques de acoplamiento y,
por lo tanto, en la 'rigidez efectiva angular' del acoplamiento
entre cada par de p\'endulos. Esto, a su vez, afecta las
caracter\'isticas de los modos normales del sistema. La
configuraci\'on 6-6, por ejemplo, debido a sus puntos de acople
espec\'ificos, facilita un tipo de interacci\'on que
favorece un modo donde el p\'endulo 3 oscila a una frecuencia mayor
que el 1, a diferencia de otras geometr\'ias de acople.
Esta sensibilidad a la posici\'on del acople tambi\'en es relevante
para interpretar la frecuencia excepcionalmente baja de
\qty{0.0021}{\Hz} previamente identificada en el p\'endulo 3 para la
configuraci\'on 5-1 con condici\'on inicial (100). En dicha
configuraci\'on, uno de los puntos de anclaje del resorte est\'a
muy pr\'oximo al pivote, lo que resulta en una 'rigidez efectiva
angular' muy d\'ebil para ciertos patrones de movimiento, lo que
consecuentemente lleva a la aparici\'on de frecuencias de
oscilaci\'on extremadamente bajas para el modo asociado.

Finalmente, es de destacar el caso de la configuraci\'on 6-4 bajo la
condici\'on inicial (100), donde ambos p\'endulos laterales (1 y 3)
presentan la misma frecuencia principal. Esta igualdad podr\'ia ser
indicativa de la excitaci\'on de un modo normal con un alto grado de
simetr\'ia en el movimiento de los p\'endulos externos.

\subsection*{Frecuencias del Modelo Teórico}

Al resolver el problema de valores propios para el sistema representado
en forma matricial, \cref{eq:matrix-form}, se obtuvieron las
frecuencias te\'oricas de los modos normales para cada una de las
cinco configuraciones experimentales. Estos resultados se presentan
en la \cref{tab:theoric-freq}. Como es de esperar para un sistema
con tres grados de libertad, se identifican tres frecuencias de modo
normal distintas para cada configuraci\'on.

Los valores te\'oricos en la \cref{tab:theoric-freq}
sugieren una tendencia general donde una mayor distancia entre el
punto de acople del resorte y el pivote del p\'endulo tiende a
correlacionarse con un aumento en las frecuencias de los modos normales.
En cuanto a los valores espec\'ificos, se observa
con regularidad una frecuencia te\'orica cercana a \qty{0.8}{\Hz},
as\'i como otras agrupadas en torno a \qty{1.3}{\Hz} y, en algunos
casos, pr\'oximas a \qty{1.6}{\Hz}.

Es importante destacar que las frecuencias te\'oricas obtenidas, en
general, no difieren significativamente de los valores experimentales
predominantes que fueron presentados en la \cref{tab:frequencies}.
La notable excepci\'on es la frecuencia experimental an\'omalamente
baja de \qty{0.0021}{\Hz} (identificada previamente en la
Configuraci\'on 5-1, condici\'on inicial (100) para el p\'endulo 3),
la cual no tiene una contraparte directa en el espectro te\'orico
calculado. No obstante, la concordancia general para las dem\'as
frecuencias principales sugiere que el modelo te\'orico linealizado,
basado en la aproximaci\'on de peque\~nas oscilaciones, logra
aproximar razonablemente el comportamiento din\'amico del sistema
experimental construido.

\begin{table*}[htbp!]
	\centering
	\caption{Frecuencias te\'oricas de los modos normales ($f_{0i}$)
		calculadas para el sistema, seg\'un la configuraci\'on de
		acoplamiento.
	}
	\label{tab:theoric-freq}
	\pgfplotstabletypeset[
	every head row/.style={
		before row=\toprule,
		after row=\midrule
	},
	every last row/.style={after row=\bottomrule},
	columns/config/.style={
		string type,
		column name={Configuración},
	},
	columns/f2/.style={
		column name=$f_{01} [\si{\Hz}]$,
		fixed,
		fixed zerofill,
		precision=3,
	},
	columns/f1/.style={
		column name=$f_{02} [\si{\Hz}]$,
		fixed,
		fixed zerofill,
		precision=3,
	},
	columns/f3/.style={
		column name=$f_{03} [\si{\Hz}]$,
		fixed,
		fixed zerofill,
		precision=3,
	},
	columns={config, f2, f1, f3}
	]{\theoric}
\end{table*}

En la \cref{tab:comparison} se presenta una comparaci\'on directa
entre las frecuencias experimentales ($f_\text{ex}$) predominantes,
identificadas para cada modo normal discernible en las distintas
configuraciones, y sus correspondientes frecuencias te\'oricas
($f_\text{te}$). La tabla tambi\'en incluye el error relativo
porcentual, calculado como
$\mid \left(1 - \frac{f_\text{te}}{f_\text{ex}}\right) \mid 100\%$,
para cuantificar la discrepancia.

Los errores relativos porcentuales var\'ian considerablemente.
El error m\'as grande observado, sin considerar la frecuencia
an\'omala de \qty{0.0021}{\Hz} que no se incluye en esta
comparaci\'on directa de modos, es del \qty{2.509}{\percent}. Por
otro lado, el error m\'as bajo registrado es de tan solo
\qty{0.036}{\percent}. Estos valores indican que, si bien existen
desviaciones, el modelo te\'orico es capaz de predecir las
frecuencias de los modos normales con un grado de precisi\'on
generalmente alto para la mayor\'ia de los casos estudiados.

\begin{table*}[htbp!]
	\centering
	\caption{Comparaci\'on entre las frecuencias experimentales
		predominantes ($f_\text{ex}$) y las frecuencias te\'oricas
		($f_\text{te}$) para los modos normales, junto con el error
		relativo porcentual.
	}
	\label{tab:comparison}
	\pgfplotstabletypeset[
	every head row/.style={
		before row=\toprule,
		after row=\midrule
	},
	every last row/.style={after row=\bottomrule},
	columns/config/.style={
		string type,
		column name={Configuraci\'on},
	},
	columns/fe/.style={
		column name=$f_\text{ex} [\si{\Hz}]$,
		fixed,
		fixed zerofill,
		precision=3,
	},
	columns/ft/.style={
		column name=$f_\text{te} [\si{\Hz}]$,
		fixed,
		fixed zerofill,
		precision=3,
	},
	columns/err/.style={
		column name={Error Relativo $[\%]$},
		fixed,
		fixed zerofill,
		precision=3,
	},
	every nth row={3}{before row=\midrule},
	columns={config, fe, ft, err}
	]{\comparison}
\end{table*}

\subsection*{Casos Representativos de Oscilaci\'on y Espectros de Frecuencia}

A partir del conjunto de datos experimentales recopilados, se
seleccionaron seis casos considerados como los m\'as representativos
de los diversos comportamientos din\'amicos observados en el sistema
de p\'endulos acoplados. Para cada uno de estos casos seleccionados,
se presentan figuras que ilustran de manera conjunta: (a) la
evoluci\'on tempo ral del desplazamiento angular $\theta_i(t)$ de los
p\'endulos involucrados y (b) el correspondiente espectro de
amplitudes obtenido a partir de la FFT de dichas series temporales.
Estas representaciones gr\'aficas se encuentran detalladas en las
\cref{fig:111-11,fig:010-51,fig:101-61,fig:010-62,fig:100-66,fig:010-66}.

\begin{figure}[htbp!]
	\centering
	\includegraphics[width=\linewidth]{./Figures/111_11_filtrado.pdf}
	\caption{Evoluci\'on temporal del desplazamiento angular y espectro de
		frecuencias (FFT) para la Configuraci\'on 1-1 con condiciones
	iniciales (111).}
	\label{fig:111-11}
\end{figure}

\begin{figure}[htbp!]
	\centering
	\includegraphics[width=\linewidth]{./Figures/010_15_filtrado.pdf}
	\caption{Evoluci\'on temporal del desplazamiento angular y espectro de
		frecuencias (FFT) para la Configuraci\'on 5-1 con condiciones
	iniciales (010).}
	\label{fig:010-51}
\end{figure}

\begin{figure}[htbp!]
	\centering
	\includegraphics[width=\linewidth]{./Figures/101_16_filtrado.pdf}
	\caption{Evoluci\'on temporal del desplazamiento angular y espectro de
		frecuencias (FFT) para la Configuraci\'on 6-1 con condiciones
	iniciales (101).}
	\label{fig:101-61}
\end{figure}

\begin{figure}[htbp!]
	\centering
	\includegraphics[width=\linewidth]{./Figures/010_26_filtrado.pdf}
	\caption{Evoluci\'on temporal del desplazamiento angular y espectro de
		frecuencias (FFT) para la Configuraci\'on 6-2 con condiciones
	iniciales (010).}
	\label{fig:010-62}
\end{figure}

\begin{figure}[htbp!]
	\centering
	\includegraphics[width=\linewidth]{./Figures/001_66_filtrado.pdf}
	\caption{Evoluci\'on temporal del desplazamiento angular y espectro de
		frecuencias (FFT) para la Configuraci\'on 6-6 con condiciones
	iniciales (100).}
	\label{fig:100-66}
\end{figure}

\begin{figure}[htbp!]
	\centering
	\includegraphics[width=\linewidth]{./Figures/010_66_filtrado.pdf}
	\caption{Evoluci\'on temporal del desplazamiento angular y espectro de
		frecuencias (FFT) para la Configuraci\'on 6-6 con condiciones
	iniciales (010).}
	\label{fig:010-66}
\end{figure}

En la \cref{fig:111-11}, correspondiente a la Configuraci\'on 1-1 con
condiciones iniciales (111), se observa que cada p\'endulo exhibe un
pico de frecuencia principal distintivo en su espectro FFT, sin
compartir una \'unica frecuencia dominante com\'un. Estos picos
espectrales son relativamente anchos, lo que sugiere una duraci\'on
efectiva de coherencia limitada para estas oscilaciones o la
presencia de un amortiguamiento significativo. Dicho amortiguamiento
es esperable debido a las interacciones inherentes entre los
p\'endulos (acoplamiento y fricci\'on) y se ve acentuado por la
complejidad del movimiento resultante de las condiciones iniciales,
donde los p\'endulos laterales se excitan desfasados respecto al
central. Particularmente, el comportamiento temporal del p\'endulo 2
en este caso se desv\'ia de un movimiento arm\'onico
simple, presentando modulaciones en su amplitud.

En contraste, la \cref{fig:010-51} (Configuraci\'on 1-5, condiciones
iniciales (010)) muestra picos espectrales considerablemente m\'as
definidos y estrechos, indicativo de un comportamiento m\'as regular
y menos amortiguado para las frecuencias dominantes. Notablemente,
los tres p\'endulos comparten la misma frecuencia principal,
reafirmando la tendencia observada para la condici\'on (010). Adem\'as
de la frecuencia fundamental, se aprecian componentes espectrales
secundarias comunes a los tres p\'endulos, aunque con diferentes
amplitudes relativas; por ejemplo, el p\'endulo 2 podr\'ia dominar en
amplitud en un arm\'onico secundario, mientras que el p\'endulo 1 lo
har\'ia en otro. En este caso, el p\'endulo 3 parece exhibir una
menor amplitud en sus componentes espectrales, lo que podr\'ia
atribuirse a una transmisi\'on de energ\'ia menos eficiente hacia \'el
debido a la geometr\'ia espec\'ifica de los puntos de acople en esta
configuraci\'on. La evoluci\'on temporal de los \'angulos para este
caso (\cref{fig:010-51}) es m\'as estable y coherente, con una
disminuci\'on gradual de las amplitudes generales debido al
amortiguamiento residual.

Para la Configuraci\'on 6-1 bajo condiciones iniciales (101),
ilustrada en la \cref{fig:101-61}, el an\'alisis espectral revela
nuevamente tres picos de frecuencia principales distintos, sugiriendo
la excitaci\'on de m\'ultiples modos normales. El p\'endulo 3 muestra
una mayor amplitud en su frecuencia principal y un pico espectral
m\'as estrecho en comparaci\'on con el p\'endulo 1. Esta
caracter\'istica podr\'ia estar relacionada con la naturaleza del
acoplamiento en esta configuraci\'on particular y el hecho de que el
p\'endulo 3 es uno de los excitados inicialmente (condici\'on (101)).

Por otro lado, el p\'endulo 2 presenta oscilaciones de amplitud muy
reducida, un comportamiento consistente con la excitaci\'on de un
modo normal donde los p\'endulos laterales se mueven de forma
predominantemente desfasada y el p\'endulo central tiene una
participaci\'on m\'inima. Adicionalmente, en la evoluci\'on temporal
se observa un intercambio de energ\'ia entre los p\'endulos 1 y 3:
leves aumentos en la amplitud de oscilaci\'on del p\'endulo 3 tienden
a coincidir con disminuciones en la amplitud del p\'endulo 1, y
viceversa. Este fen\'omeno es caracter\'istico de las pulsaciones
(batidos) en sistemas acoplados y es esperable si las frecuencias
de los modos que involucran principalmente a estos dos p\'endulos son
cercanas y hay una transferencia de energ\'ia entre ellos, en un
contexto de conservaci\'on aproximada de la energ\'ia (descontando
p\'erdidas por fricci\'on).

Para la configuraci\'on 6-2 bajo la condici\'on inicial (010),
ilustrada en la \cref{fig:010-62} (no incluida en el presente
documento, pero referenciada para an\'alisis), el espectro del
p\'endulo 2 muestra una frecuencia principal con una amplitud
prominente y un pico espectral estrecho en comparaci\'on con otras
componentes secundarias. Entre estas \'ultimas, se identifica un pico
cuya frecuencia es aproximadamente cuatro veces la frecuencia
principal, aunque su contribuci\'on en amplitud es mucho menor.
Debido a esta marcada dominancia de la frecuencia fundamental, el
comportamiento temporal del p\'endulo 2 se asemeja de forma notable
a un movimiento arm\'onico simple. En cuanto al p\'endulo 1, este
comparte la frecuencia principal observada en el p\'endulo 2; sin
embargo, su espectro tambi\'en exhibe otro pico significativo a una
frecuencia ligeramente superior al triple de la fundamental. Esta
composici\'on multifrecuencial indica que el movimiento del p\'endulo
1 no es estrictamente arm\'onico simple, aunque s\'i manifiesta una
clara periodicidad, como se evidencia en su evoluci\'on temporal.
Finalmente, el p\'endulo 3 presenta en su espectro dos picos con
amplitudes comparables, cuyas frecuencias se corresponden
aproximadamente con los valores te\'oricos esperados de \qty{0.83}{\Hz}
y \qty{1.2}{\Hz} para los modos involucrados. No obstante, las
amplitudes de estos picos para el p\'endulo 3 son relativamente
bajas, lo que se traduce en una oscilaci\'on de escasa magnitud para
este p\'endulo en particular.

Por \'ultimo, se analiza la configuraci\'on 6-6 con la condici\'on
inicial (100) (excitaci\'on inicial \'unicamente del p\'endulo 1).
En este caso, se observa c\'omo la geometr\'ia de acople facilita una
efectiva transmisi\'on de energ\'ia entre los p\'endulos. El an\'alisis
espectral revela que los tres p\'endulos comparten componentes de
frecuencia en las mismas ubicaciones espectrales. Espec\'ificamente,
los p\'endulos 1 y 3 comparten una frecuencia principal com\'un (m\'as
alta), mientras que el p\'endulo 2 tiene su propia frecuencia
principal a un valor menor. A pesar de ello, el p\'endulo 2 tambi\'en
exhibe picos secundarios en las frecuencias donde los p\'endulos
laterales oscilan predominantemente, indicando su participaci\'on en
esos modos de mayor frecuencia. Este comportamiento, donde todos los
p\'endulos oscilan de manera significativa, distingue a esta
combinaci\'on de configuraci\'on y condici\'on inicial de otras
pruebas hipot\'eticas con la misma condici\'on (100) pero diferentes
puntos de acople, resaltando as\'i el papel determinante de la
posici\'on de los acoples en la din\'amica del sistema.

\section{Conclusiones}

Este estudio ha investigado con \'exito el comportamiento din\'amico
de un sistema de tres p\'endulos acoplados, confirmando la validez y
aplicabilidad de la aproximaci\'on de peque\~nas oscilaciones para
describir su movimiento. Mediante un montaje experimental controlado y
el an\'alisis de datos con la Transformada R\'apida de Fourier (FFT),
se lograron identificar las frecuencias de los modos normales de
oscilaci\'on, cumpliendo el objetivo principal de visualizar y
cuantificar estos patrones fundamentales en un sistema cl\'asico.

Los resultados experimentales mostraron una rica fenomenolog\'ia de
oscilaciones acopladas, donde la frecuencia dominante de cada p\'endulo
var\'ia seg\'un la configuraci\'on de los resortes y las condiciones
iniciales aplicadas. El p\'endulo central (el m\'as largo) exhibi\'o
una frecuencia notablemente consistente alrededor de \qty{0.844}{\Hz}.
Por su parte, los p\'endulos laterales a menudo presentaron una
dualidad de frecuencias predominantes, lo que sugiere la excitaci\'on
simult\'anea de m\'ultiples modos. La comparaci\'on cuantitativa revel\'o
que el modelo te\'orico linealizado predice estas frecuencias con
una precisi\'on notable, con errores relativos porcentuales que en
la mayor\'ia de los casos se sit\'uan entre \qty{0.036}{\percent} y
\qty{2.509}{\percent}. Esta estrecha concordancia valida la capacidad
del marco te\'orico para describir con fiabilidad el sistema
experimental.

El an\'alisis tambi\'en puso de manifiesto el papel cr\'itico de la
geometr\'ia de acoplamiento. Se observ\'o c\'omo la posici\'on de los
resortes influye en la rigidez efectiva angular, determinando la
transmisi\'on de energ\'ia y las frecuencias de los modos excitados.
Particularmente, ciertas configuraciones y condiciones iniciales
favorecieron la oscilaci\'on de los tres p\'endulos a una misma
frecuencia, mientras que otras revelaron complejas interacciones de
energ\'ia, como el fen\'omeno de pulsaciones entre los p\'endulos
laterales.

Sin embargo, el estudio tambi\'en identific\'o una frecuencia
experimental an\'omala excepcionalmente baja (\qty{0.0021}{\Hz}) en
una configuraci\'on espec\'ifica, la cual no encuentra una
contraparte directa en el modelo te\'orico linealizado. Esta
discrepancia sugiere la posible influencia de factores no incluidos
en el modelo ideal, como efectos no lineales para desplazamientos
angulares mayores, o la presencia de amortiguamiento diferencial en
el sistema.

Como trabajo futuro, ser\'ia valioso: (1) incorporar t\'erminos de
amortiguamiento en el modelo te\'orico para una comparaci\'on m\'as
precisa con los datos experimentales, (2) realizar un an\'alisis
cuantitativo detallado de los vectores propios (formas modales) para
relacionar directamente los patrones de movimiento observados con los
modos te\'oricos, y (3) explorar el comportamiento del sistema bajo
condiciones que acent\'uen las no linealidades, con el fin de entender
fen\'omenos como la frecuencia an\'omala observada. Este estudio no
solo consolida la comprensi\'on de las oscilaciones cl\'asicas
acopladas, sino que tambi\'en sienta las bases conceptuales para
abordar sistemas m\'as complejos en la mec\'anica cu\'antica.


\bibliography{ref}

\end{document}
