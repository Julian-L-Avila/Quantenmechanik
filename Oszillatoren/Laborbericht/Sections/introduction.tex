\section{Introducción}

En el estudio de la mec\'anica cl\'asica, la comprensi\'on del
comportamiento din\'amico de los sistemas es fundamental.
Particularmente, los sistemas oscilatorios juegan un papel
crucial en una amplia gama de fen\'omenos, desde el movimiento
de un p\'endulo simple hasta las vibraciones de estructuras
complejas.
A menudo, estos sistemas involucran m\'ultiples componentes que interact\'uan
entre s\'i, dando lugar a un comportamiento acoplado que es m\'as complejo que
la suma de sus partes individuales.

Una herramienta anal\'itica extraordinariamente potente para
desentra\~nar esta complejidad es la aproximaci\'on
de peque\~nas oscilaciones.
Esta aproximaci\'on establece que cualquier sistema f\'isico, bajo ciertas
condiciones y cerca de una posici\'on de equilibrio estable, puede ser
linealizado.
Dicho de otro modo, el sistema se comporta de manera an\'aloga a un conjunto de
osciladores arm\'onicos simples acoplados.
La fuerza de esta aproximaci\'on reside en que permite transformar problemas
intrincados en un marco matem\'atico m\'as manejable, revelando las frecuencias
naturales y los modos normales de oscilaci\'on del sistema.

Los modos normales representan patrones de movimiento colectivo en los que todas
las partes del sistema oscilan con la misma frecuencia y fase relativa constante.
Visualizar y comprender estos modos es esencial, ya que ofrecen una perspectiva
fundamental sobre la estabilidad y la respuesta din\'amica de un sistema.
En mec\'anica cl\'asica, los modos normales son la clave para entender fen\'omenos
como la resonancia, la propagaci\'on de ondas en medios continuos y la
transferencia de energ\'ia entre osciladores \cite{Goldstein}.

Este concepto trasciende la f\'isica cl\'asica y sienta las bases para la
comprensi\'on de fen\'omenos cu\'anticos.
La idea de estados propios discretos y sus correspondientes valores propios en
sistemas cu\'anticos tiene un paralelismo conceptual directo con los modos
normales y las frecuencias naturales en la mec\'anica cl\'asica.
Es, por tanto, un primer paso indispensable para adentrarse en la mec\'anica
cu\'antica y la comprensi\'on de sistemas de m\'ultiples part\'iculas
\cite{Griffiths2017}.

En este trabajo, se presenta un sistema de tres p\'endulos acoplados dise\~nado
espec\'ificamente para visualizar y estudiar experimentalmente sus modos normales
de oscilaci\'on.
A trav\'es de este sistema, se busca proporcionar una plataforma que permita
explorar directamente los principios de la aproximaci\'on de peque\~nas
oscilaciones, la determinaci\'on de frecuencias naturales y la identificaci\'on
de los modos normales.
