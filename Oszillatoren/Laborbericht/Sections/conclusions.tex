\section{Conclusiones}

Este estudio ha investigado con \'exito el comportamiento din\'amico
de un sistema de tres p\'endulos acoplados, confirmando la validez y
aplicabilidad de la aproximaci\'on de peque\~nas oscilaciones para
describir su movimiento. Mediante un montaje experimental controlado y
el an\'alisis de datos con la Transformada R\'apida de Fourier (FFT),
se lograron identificar las frecuencias de los modos normales de
oscilaci\'on, cumpliendo el objetivo principal de visualizar y
cuantificar estos patrones fundamentales en un sistema cl\'asico.

Los resultados experimentales mostraron una rica fenomenolog\'ia de
oscilaciones acopladas, donde la frecuencia dominante de cada p\'endulo
var\'ia seg\'un la configuraci\'on de los resortes y las condiciones
iniciales aplicadas. El p\'endulo central (el m\'as largo) exhibi\'o
una frecuencia notablemente consistente alrededor de \qty{0.844}{\Hz}.
Por su parte, los p\'endulos laterales a menudo presentaron una
dualidad de frecuencias predominantes, lo que sugiere la excitaci\'on
simult\'anea de m\'ultiples modos. La comparaci\'on cuantitativa revel\'o
que el modelo te\'orico linealizado predice estas frecuencias con
una precisi\'on notable, con errores relativos porcentuales que en
la mayor\'ia de los casos se sit\'uan entre \qty{0.036}{\percent} y
\qty{2.509}{\percent}. Esta estrecha concordancia valida la capacidad
del marco te\'orico para describir con fiabilidad el sistema
experimental.

El an\'alisis tambi\'en puso de manifiesto el papel cr\'itico de la
geometr\'ia de acoplamiento. Se observ\'o c\'omo la posici\'on de los
resortes influye en la rigidez efectiva angular, determinando la
transmisi\'on de energ\'ia y las frecuencias de los modos excitados.
Particularmente, ciertas configuraciones y condiciones iniciales
favorecieron la oscilaci\'on de los tres p\'endulos a una misma
frecuencia, mientras que otras revelaron complejas interacciones de
energ\'ia, como el fen\'omeno de pulsaciones entre los p\'endulos
laterales.

Sin embargo, el estudio tambi\'en identific\'o una frecuencia
experimental an\'omala excepcionalmente baja (\qty{0.0021}{\Hz}) en
una configuraci\'on espec\'ifica, la cual no encuentra una
contraparte directa en el modelo te\'orico linealizado. Esta
discrepancia sugiere la posible influencia de factores no incluidos
en el modelo ideal, como efectos no lineales para desplazamientos
angulares mayores, o la presencia de amortiguamiento diferencial en
el sistema.

Como trabajo futuro, ser\'ia valioso: (1) incorporar t\'erminos de
amortiguamiento en el modelo te\'orico para una comparaci\'on m\'as
precisa con los datos experimentales, (2) realizar un an\'alisis
cuantitativo detallado de los vectores propios (formas modales) para
relacionar directamente los patrones de movimiento observados con los
modos te\'oricos, y (3) explorar el comportamiento del sistema bajo
condiciones que acent\'uen las no linealidades, con el fin de entender
fen\'omenos como la frecuencia an\'omala observada. Este estudio no
solo consolida la comprensi\'on de las oscilaciones cl\'asicas
acopladas, sino que tambi\'en sienta las bases conceptuales para
abordar sistemas m\'as complejos en la mec\'anica cu\'antica.
