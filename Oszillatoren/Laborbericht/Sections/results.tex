\section{Resultados y An\'alisis}

Tras el montaje del sistema de p\'endulos acoplados y el registro
de las mediciones mediante el sensor \emph{Cassy}, se recopil\'o un
conjunto de 15 series de datos. Estas series corresponden a tres
condiciones iniciales distintas para cada una de las cinco
configuraciones experimentales estudiadas. Los datos temporales de
los \'angulos de cada p\'endulo fueron procesados utilizando un
c\'odigo en Python, con el fin de generar gr\'aficas de la evoluci\'on
angular $\theta_i(t)$ y, fundamentalmente, para determinar las
frecuencias angulares principales de oscilaci\'on mediante la
aplicaci\'on de la Transformada R\'apida de Fourier (FFT).

En la \cref{tab:frequencies} se presenta un resumen de las
frecuencias angulares principales identificadas para cada p\'endulo,
en funci\'on de la configuraci\'on del sistema y de la condici\'on
inicial aplicada. Un an\'alisis preliminar de estos valores revela
patrones interesantes: la frecuencia angular usual del p\'endulo 2
(el m\'as largo) se sit\'ua consistentemente alrededor de
\qty{0.844}{\radian\Hz}, con una desviaci\'on est\'andar
reducida de \num{0.009}, lo que indica una notable regularidad en su
comportamiento oscilatorio a esta frecuencia.

En contraste, para el p\'endulo 1, se identifican dos agrupaciones
principales de frecuencias angulares: una en torno a
\qty{1.311}{\radian\Hz} (desviaci\'on est\'andar de \num{0.019})
y otra cercana a \qty{0.843}{\radian\Hz} (desviaci\'on
est\'andar de \num{0.008}). Un comportamiento similar se observa en el
p\'endulo 3, el cual exhibe frecuencias angulares predominantes
alrededor de \qty{1.32}{\radian\Hz} (desviaci\'on est\'andar
de \num{0.020}) y \qty{0.843}{\radian\Hz} (desviaci\'on
est\'andar de \num{0.008}). Esta dualidad en las frecuencias
dominantes para los p\'endulos 1 y 3 sugiere la excitaci\'on
selectiva de diferentes modos normales del sistema, dependiendo de la
configuraci\'on de acoplamiento y las condiciones iniciales.

\begin{table*}[htbp!]
	\centering
	\caption{Frecuencias angulares principales de oscilaci\'on
		($\omega_i$) identificadas para cada p\'endulo, seg\'un la
		configuraci\'on experimental y las condiciones iniciales aplicadas.
	}
	\label{tab:frequencies}
	\pgfplotstabletypeset[
	every head row/.style={
		before row=\toprule,
		after row=\midrule
	},
	every last row/.style={after row=\bottomrule},
	columns/config/.style={
		string type,
		column name={Configuración},
	},
	columns/mode/.style={
		string type,
		column name={Condición Inicial},
	},
	columns/freq1/.style={
		column name=$\omega_1 [\si{\radian\Hz}]$,
		fixed,
		fixed zerofill,
		precision=3,
	},
	columns/freq2/.style={
		column name=$\omega_2 [\si{\radian\Hz}]$,
		fixed,
		fixed zerofill,
		precision=3,
	},
	columns/freq3/.style={
		column name=$\omega_3 [\si{\radian\Hz}]$,
		fixed,
		fixed zerofill,
		precision=3,
	},
	every nth row={3}{before row=\midrule}
	]{\datafreq}
\end{table*}
