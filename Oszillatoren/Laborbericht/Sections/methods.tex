\section{Ecuaciones de Movimiento}

Para describir la din\'amica del sistema compuesto por tres p\'endulos
f\'isicos, se establecen consideraciones iniciales que simplifican el
an\'alisis y la deducci\'on de las ecuaciones de movimiento.
Inicialmente, los p\'endulos se representan en sus puntos de
equilibrio, donde los ejes de rotaci\'on est\'an separados por una
distancia $a$, correspondiente a la longitud natural de los muelles.
Este esquema se ilustra en la \cref{fig:esquema_equilibrio}.

\begin{figure}[htbp!]
  \centering
  \includegraphics[width=0.8\linewidth]{./Figures/system-diagram.pdf}
  \caption{Esquema de los p\'endulos en equilibrio. La distancia
  entre los ejes de rotaci\'on es $a$.}
  \label{fig:esquema_equilibrio}
\end{figure}

Las ecuaciones de movimiento del sistema se obtienen mediante la
aplicaci\'on de la sumatoria de torques, considerando tanto las
contribuciones de los resortes como la de la fuerza gravitacional
. Si bien los formalismos Lagrangiano y Hamiltoniano
ofrecen un enfoque m\'as general y elegante para sistemas complejos,
para el presente sistema de p\'endulos, la aplicaci\'on directa de
la segunda ley de Newton para rotaci\'on (sumatoria de torques)
resulta en un m\'etodo igualmente riguroso y conceptualmente claro
para la derivaci\'on de las ecuaciones de movimiento.
Es necesario, entonces, encontrar una expresi\'on para el torque
gravitacional en funci\'on de cualquier \'angulo $\theta_i$, as\'i
como una expresi\'on para las distancias del tipo $x_0 + \Delta x$,
que representan la elongaci\'on de los resortes en funci\'on de los
\'angulos $\theta_i$ y $\theta_{i\pm 1}$. Las barras se
enumeran de izquierda a derecha como 1, 2 y 3, seg\'un se muestra en la
\cref{fig:enumeracion_barras}.

\begin{figure}[htbp!]
  \centering
  \includegraphics[width=0.8\linewidth]{Figures/Ilustración_sin_título 4.pdf}
  \caption{Numeraci\'on de las barras del sistema de p\'endulos:
  1 (izquierda), 2 (central), 3 (derecha).}
  \label{fig:enumeracion_barras}
\end{figure}

Al desplazar las barras 1 y 2 de su posici\'on de equilibrio, se
observa que la tangente de cada barra se desv\'ia un \'angulo
$\theta_i$ del eje horizontal. Esta condici\'on permite
construir un tri\'angulo rect\'angulo con las componentes $x_i$ e
$y_i$ de la posici\'on del punto de aplicaci\'on de la fuerza
el\'astica para cada barra, generalizada a cualquier fracci\'on de
la longitud $l$, tal como se ilustra en la \cref{fig:triangulo_posicion}
.

\begin{figure}[htbp!]
  \centering
  \includegraphics[width=0.8\linewidth]{Figures/Ilustración_sin_título 5.pdf}
  \caption{Construcci\'on geom\'etrica para determinar la posici\'on
  del punto de aplicaci\'on de la fuerza el\'astica en una barra.}
  \label{fig:triangulo_posicion}
\end{figure}

Las coordenadas $x_k$ e $y_k$ para $k = 1, 2$ est\'an definidas como:
\begin{align}
  x_k &= c l \sin(\theta_{k}) \label{eq:xk} \\
  y_k &= c l \cos(\theta_{k}) \label{eq:yk}
\end{align}
donde $c$ y $d$ son fracciones de la longitud de la barra $l$ que
indican la posici\'on de los puntos de acople de los resortes.
El tri\'angulo resultante est\'a definido por un \'angulo $\psi_1$,
el cual cumple la relaci\'on $\beta_1 + \psi_1 = \theta_1$, donde
$\beta_1$ corresponde al \'angulo entre la tangente de la barra 1 y
la l\'inea recta que conecta los puntos de aplicaci\'on.
A partir de este tri\'angulo, se obtiene la expresi\'on para la
elongaci\'on del resorte entre las barras 1 y 2:
\begin{equation}
  x_{01} + \Delta x_{1} = \sqrt{(y_2 - y_1)^2 + (a + x_2 - x_1)^2}
  \label{eq:elongacion1}
\end{equation}
con $x_{01}$ como la longitud natural del resorte.
Y el \'angulo $\psi_1$ est\'a dado por:
\begin{equation}
  \psi_1 = \arctan\left( \frac{y_2 - y_1}{a - x_1 + x_2} \right)
  \label{eq:psi1}
\end{equation}
De forma an\'aloga, para el resorte entre las barras 2 y 3, se
utiliza un punto de aplicaci\'on $d l$, resultando en:
\begin{align}
  \psi_2 &= \arctan \left( \frac{y_4 - y_3}{a - x_3 + x_4} \right) \label{eq:psi2}, \\
  x_{02} + \Delta x_{2} &= \sqrt{(y_4 - y_3)^2 + (a + x_4 - x_3)^2} \label{eq:elongacion2}
\end{align}
Donde para $j = 3, 4$, las coordenadas $x_j$ e $y_j$ son:
\begin{align}
  x_j &= d l \sin(\theta_{j}) \label{eq:xj} \\
  y_j &= d l \cos(\theta_{j}) \label{eq:yj}
\end{align}
El torque gravitacional sobre cada barra $i$ est\'a dado por:
\begin{equation}
  \tau_{g_i} = y_{\text{cm}_i} \, m_i \, g \sin(\theta_i) \label{eq:tau_g}
\end{equation}
Donde $y_{\text{cm}_i}$ es la posici\'on del centro de masa de la
barra $i$. Para los torques el\'asticos, se deben
considerar los \'angulos $\beta_i$, definidos como:
\begin{align}
  \beta_1 + \psi_1 &= \theta_1, & \beta_2 + \psi_1 &= \theta_2, \label{eq:beta1_psi1}\\
  \beta_3 + \psi_2 &= \theta_2, & \beta_4 + \psi_2 &= \theta_3 \label{eq:beta3_psi2}
\end{align}
Es importante notar que, debido a las distintas posiciones de
aplicaci\'on de los resortes entre los pares de barras, aparecen
cuatro \'angulos $\beta_i$, donde los \'indices 2 y 3 se refieren
a la barra central. La sumatoria de torques para cada
p\'endulo, considerando el sentido antihorario como positivo, resulta
en las siguientes ecuaciones de movimiento:
\begin{align}
  \sum \tau_{m_1} &= c l k_1 (\Delta x_1) \cos(\beta_1) - m_1 g y_{\text{cm}_1} \sin(\theta_1) = I_1 \, \ddot{\theta}_1 \label{eq:tau_m1}, \\
  \sum \tau_{m_2} &= d l k_2 (\Delta x_2) \cos(\beta_3) - c l k_1 (\Delta x_1) \cos(\beta_2) - m_2 g y_{\text{cm}_2} \sin(\theta_2) = I_2 \, \ddot{\theta}_2 \label{eq:tau_m2}, \\
  \sum \tau_{m_3} &= -d l k_2 (\Delta x_2) \cos(\beta_4) + m_3 g y_{\text{cm}_3} \sin(\theta_3) = I_3 \, \ddot{\theta}_3 \label{eq:tau_m3}
\end{align}

\subsection*{Aproximaciones para \'Angulos Peque\~nos}

Con el fin de obtener una soluci\'on anal\'itica aproximada que permita
calcular las frecuencias naturales del sistema, se aplican
las siguientes aproximaciones para \'angulos peque\~nos ($\theta \ll 1$ rad):
\begin{align*}
  \sin(\theta) &\approx \theta, \\
  \cos(\theta) &\approx 1, \\
  \tan(\theta) &\approx \theta.
\end{align*}
Aplicando estas aproximaciones, las coordenadas $x_k$ e $y_k$ se
simplifican a:
\begin{align*}
  x_1 &= c l \theta_1, & x_2 &= c l \theta_2 \\
  x_3 &= d l \theta_2, & x_4 &= d l \theta_3 \\
  y_1 &= c l, & y_2 &= c l \\
  y_3 &= d l, & y_4 &= d l
\end{align*}
Si se cumple la condici\'on de que las longitudes naturales de los
resortes son iguales a la separaci\'on entre los ejes, es decir,
$x_{01} = x_{02} = a$, las elongaciones de los resortes $\Delta x_1$
y $\Delta x_2$ se simplifican a:
\begin{align}
  \Delta x_1 &= cl(\theta_2 - \theta_1) \label{eq:deltax1_approx}, \\
  \Delta x_2 &= dl(\theta_3 - \theta_2) \label{eq:deltax2_approx}.
\end{align}
Adem\'as, si se asume que $\arctan(\psi_i) \approx \psi_i \approx 0$,
lo que implica que la fuerza el\'astica act\'ua aproximadamente en
la direcci\'on horizontal, entonces los \'angulos $\beta_i$ se
aproximan a los \'angulos $\theta_i$ ($\beta_i \approx \theta_i$).
Esto simplifica la sumatoria de torques a:
\begin{align}
  I_1 \ddot{\theta}_1 &= (cl)^2 k_1 (\theta_2 - \theta_1) - m_1 g y_{\text{cm}_1} \theta_1 \label{eq:eq_mov_lin1}, \\
  I_2 \ddot{\theta}_2 &= (dl)^2 k_2 (\theta_3 - \theta_2) - (cl)^2 k_1 (\theta_2 - \theta_1) - m_2 g y_{\text{cm}_2} \theta_2 \label{eq:eq_mov_lin2}, \\
  I_3 \ddot{\theta}_3 &= -(dl)^2 k_2 (\theta_3 - \theta_2) + m_3 g y_{\text{cm}_3} \theta_3 \label{eq:eq_mov_lin3}.
\end{align}
Finalmente, el sistema de ecuaciones diferenciales lineales queda
expresado como:
\begin{align}
  \ddot{\theta}_1 &=\; \theta_1 \left( \frac{(cl)^2 k_{1} - y_{\text{cm}_1} m_1 g}{I_1} \right) + \theta_2 \left( -\frac{k_1 (cl)^2}{I_1} \right) \label{eq:eom1} \\
  \ddot{\theta}_2 &=\; \theta_1 \left( \frac{k_1 (cl)^2}{I_2} \right) + \theta_2 \left( \frac{k_2 (dl)^2 - k_1 (cl)^2 + y_{\text{cm}_2} m_2 g}{I_2} \right) + \theta_3 \left( -\frac{k_2 (dl)^2}{I_2} \right) \label{eq:eom2} \\
  \ddot{\theta}_3 &=\; \theta_2 \left( -\frac{k_2 (dl)^2}{I_3} \right) + \theta_3 \left( \frac{k_2 (dl)^2 - y_{\text{cm}_3} m_3 g}{I_3} \right) \label{eq:eom3}
\end{align}
Estas ecuaciones forman un sistema lineal de ecuaciones
diferenciales ordinarias acopladas, que pueden ser presentadas
como una ecuaci\'on matricial \cref{eq:matrix-form} donde
$\bm{M}$, $\bm{K}$ y $\bm{\Theta}$ representan una matriz cuadrada
diagonal con los t\'erminos de inercia, una matriz cuadrada con los
t\'erminos de acople, y un vector columna con las coordenadas
angulares, respectivamente.
\begin{equation}
  \mathbf{M} \ddot{\bm{\Theta}} + \mathbf{K} \bm{\Theta} = \mathbf{0}
  \label{eq:matrix-form}
\end{equation}
Este sistema linealizado permite determinar las frecuencias
naturales del sistema al resolver el problema de valores propios
asociado.

\subsection*{Modos normales}

A partir de la ecuaci\'on en forma matricial \cref{eq:matrix-form} se
lleva a un problema de valores propios para determinar los modos
normales. Esto se logra con la posibilidad de invertir la matriz con
los t\'erminos inerciales e identificar la segunda derivada como el
operador:
\begin{equation}
  \mathbf{D}^2_t \bm{\Theta} = - \mathbf{\Omega} \bm{\Theta}
  \label{eq:eigenproblem}
\end{equation}
donde $\mathbf{D}^2_t$ representa el operador de la segunda derivada
temporal y $\mathbf{\Omega} = \mathbf{M}^{-1}\mathbf{K}$ es la matriz
din\'amica del sistema. Dada la complejidad del c\'alculo anal\'itico
para un sistema de tres grados de libertad, se emplea una soluci\'on
num\'erica usando la librer\'ia de Python SciPy para determinar
las frecuencias de los modos normales y sus correspondientes vectores
propios.

\section{Metodolog\'ia Experimental}

La metodolog\'ia experimental se estructur\'o en varias etapas
fundamentales: dise\~no y construcci\'on del montaje,
caracterizaci\'on de sus componentes y adquisici\'on de datos
mediante sensores angulares.

\subsection{Construcci\'on del Montaje}

Para la construcci\'on del sistema se utilizaron tres barras
met\'alicas de alta densidad, lo que contribuye a la rigidez del
sistema y a una mejor definici\'on de su centro de masa.
Las barras laterales tienen una longitud de
$l/2 = \qty{28.0(1)}{\centi\metre}$, mientras que la barra central
posee una longitud de $l = \qty{56.0(1)}{\centi\metre}$.
Las masas y las posiciones de los centros de masa de cada barra
fueron determinadas experimentalmente y se resumen en la
\cref{tab:bars-dat}. La posici\'on del centro de masa,
$y_{\text{cm},i}$, se mide desde el punto de pivote de cada p\'endulo.

\begin{table}[htbp!]
  \caption{Parámetros físicos de las barras empleadas en el montaje.La incertidumbre para la posición del centro de masa (\(y_{\text{cm}_i}\)) es de \qty{0.1}{\centi\metre} y para la masa (\(m_i\)) es de \qty{0.1}{\gram}.}
  \centering
  \pgfplotstabletypeset[
  col sep=comma,
  zerofill,
  columns/i/.style={
    string type,
    column type={c},
    column name={\(i\)},
  },
  columns/y_cm_i/.style={
    column name={\(y_{\text{cm}_i} [\si{\centi\metre}]\)},
    precision=1,
    fixed,
    fixed zerofill,
  },
  columns/m_i/.style={
    column type={c},
    column name={\(m_i [\si{\gram}]\)},
    dec sep align,
    precision=1,
    fixed,
    fixed zerofill,
  },
  every head row/.style={
    before row=\toprule,
    after row=\midrule,
  },
  every last row/.style={
    after row=\bottomrule,
  }
  ]\mydata
  \label{tab:bars-dat}
\end{table}

Las barras fueron perforadas en m\'ultiples puntos para permitir
diversas configuraciones de acople de los resortes. La
\cref{fig:barras} muestra las barras met\'alicas preparadas para el
montaje.

\begin{figure}[htbp!]
  \centering
  \includegraphics[width=0.6\linewidth]{Figures/metal-bars.jpeg}
  \caption{Barras met\'alicas preparadas para el montaje de los
  p\'endulos, mostrando las perforaciones para el acople de resortes.}
  \label{fig:barras}
\end{figure}

\subsection{Medici\'on de la Constante El\'astica}

La constante el\'astica de los resortes empleados ($k_1$ y $k_2$) se
determin\'o experimentalmente. Se aplicaron masas conocidas a cada
resorte y se registraron los desplazamientos resultantes.
Mediante una regresi\'on lineal de los datos de fuerza
(peso aplicado) versus elongaci\'on, se obtuvieron los siguientes
valores para las constantes el\'asticas:
\begin{itemize}
  \item $k_1 = \qty{3.04(4)}{\N\per\m}$
  \item $k_2 = \qty{3.32(6)}{\N\per\m}$
\end{itemize}
El proceso gr\'afico de determinaci\'on de estas constantes se
ilustra en la \cref{fig:regresion}.

\begin{figure}[htbp!]
  \centering
  \includegraphics[width=0.75\linewidth]{Figures/springs-plot.pdf}
  \caption{Determinaci\'on de la constante el\'astica de los resortes
    mediante regresi\'on lineal de los datos de fuerza aplicada en
  funci\'on del desplazamiento.}
  \label{fig:regresion}
\end{figure}

\subsection{Integraci\'on del Sistema de Medici\'on}

Para registrar los desplazamientos angulares $\theta_i(t)$ de
cada p\'endulo, se integraron sensores angulares rotacionales
(Cassy) en cada uno de los puntos de pivote. Las barras de los
p\'endulos se fijaron a las poleas de los sensores utilizando
alambre dulce. Aunque esta metodolog\'ia de fijaci\'on podr\'ia
introducir un m\'inimo juego mec\'anico, se realiz\'o con cautela
para minimizar cualquier holgura y asegurar mediciones angulares
precisas. La \cref{fig:montaje} presenta el montaje experimental
completo con los sensores integrados.

\begin{figure}[htbp!]
  \centering
  \includegraphics[width=0.75\textwidth]{Figures/set-up.jpeg}
  \caption{Montaje experimental completo del sistema de tres
    p\'endulos acoplados, con los sensores angulares Cassy
  integrados en los pivotes.}
  \label{fig:montaje}
\end{figure}

\subsection{Toma de Datos y Configuraciones Experimentales}

Se realizaron mediciones bajo cinco configuraciones distintas de
acoplamiento de resortes, esquematizadas en la \cref{fig:configs}.
Para cada configuraci\'on, se investigaron tres tipos de
condiciones iniciales para excitar el sistema:
\begin{itemize}
  \item (001): Desplazamiento inicial \'unicamente del p\'endulo 3.
  \item (010): Desplazamiento inicial \'unicamente del p\'endulo 2.
  \item (101): Desplazamiento inicial sim\'etrico de los p\'endulos
    1 y 3 (en la misma direcci\'on y amplitud).
\end{itemize}
Cada medici\'on se registr\'o durante un intervalo aproximado de
\qty{30}{\second}, permitiendo la captura de m\'ultiples
oscilaciones completas. Los datos de los \'angulos en funci\'on
del tiempo fueron almacenados digitalmente para su posterior
an\'alisis y comparaci\'on con los resultados te\'oricos derivados
de los modos normales de oscilaci\'on.

\begin{figure}[htbp!]
  \centering
  \begin{subfigure}[b]{0.3\textwidth}
    \centering
    \includegraphics[width=\linewidth]{./Figures/15.pdf}
    \caption{Configuraci\'on 1-5}
    \label{fig:conf-1-5}
  \end{subfigure}
  \hfill
  \begin{subfigure}[b]{0.3\textwidth}
    \centering
    \includegraphics[width=\linewidth]{./Figures/16.pdf}
    \caption{Configuraci\'on 1-6}
    \label{fig:conf-1-6}
  \end{subfigure}
  \hfill
  \begin{subfigure}[b]{0.3\textwidth}
    \centering
    \includegraphics[width=\linewidth]{./Figures/26.pdf}
    \caption{Configuraci\'on 2-6}
    \label{fig:conf-2-6}
  \end{subfigure}

  \vspace{0.5cm} % Espacio vertical entre filas de subfiguras

  \begin{subfigure}[b]{0.45\textwidth} % Ajustado para centrar mejor dos figuras
    \centering
    \includegraphics[width=0.6\linewidth]{./Figures/46.pdf} % Ajustado para que no sea tan grande
    \caption{Configuraci\'on 4-6}
    \label{fig:conf-4-6}
  \end{subfigure}
  \hfill % Para espaciar igualmente si hubiera otra
  \begin{subfigure}[b]{0.45\textwidth} % Ajustado para centrar mejor dos figuras
    \centering
    \includegraphics[width=0.6\linewidth]{./Figures/66.pdf} % Ajustado para que no sea tan grande
    \caption{Configuraci\'on 6-6}
    \label{fig:conf-6-6}
  \end{subfigure}

  \caption{Representaci\'on esquem\'atica de las cinco configuraciones de
    acoplamiento de resortes estudiadas. La nomenclatura 'X-Y' en
    cada subfigura (e.g., 1-5) indica los orificios espec\'ificos
    (numerados) en los p\'endulos adyacentes donde se anclaron
  los extremos de los resortes.}
  \label{fig:configs}
\end{figure}
