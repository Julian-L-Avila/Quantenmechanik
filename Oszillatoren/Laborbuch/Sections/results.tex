\pgfplotstableread[col sep=comma, header=true]{
  i, y_cm_i, m_i
  1, 14.2, 600.8
  2, 28.0, 1216.3
  3, 14.0, 601.8
}\mydata

\pgfplotstableread[col sep=comma, header=true]{
  config, mode, freq1, freq2, freq3
  \multirow{3}{*}{5-1}, 100, 1.2715, 0.8408, 0.0205
  , 010, 0.8308, 0.8308, 0.8308
  , 101, 1.2771, 0.8375, 1.1934
  \multirow{3}{*}{6-1}, 100, 1.3147, 0.8383, 1.2003
  , 010, 0.8415, 0.8415, 0.8415
  , 101, 1.3144, 0.8348, 1.1901
  \multirow{3}{*}{6-2}, 100, 1.3303, 0.8455, 1.2288
  , 010, 0.8378, 0.8378, 0.8378
  , 101, 1.3189, 0.8402, 1.2114
  \multirow{3}{*}{6-4}, 100, 1.3263, 0.8547, 1.3263
  , 010, 0.8498, 0.8498, 0.8498
  , 101, 1.3250, 0.8505, 1.2713
  \multirow{3}{*}{6-6}, 100, 1.3144, 0.8552, 1.3303
  , 010, 0.8519, 0.8519, 0.8519
  , 101, 1.3147, 0.8574, 1.3474
}\datafreq

\pgfplotstableread[col sep=comma, header=true]{
  config, f1, f2, f3
  5-1, 1.1669, 0.8305, 1.2647
  6-1, 1.1669, 0.8343, 1.3120
  6-2, 1.1815, 0.8366, 1.3120
  6-4, 1.2394, 0.8437, 1.3122
  6-6, 1.3100, 0.8513, 1.3356
}\theoric

\pgfplotstableread[col sep=comma, header=true]{
  config, fe, ft, err
  \multirow{3}{*}{5-1}, 0.8308, 0.8305, 0.0361
  , 1.1934, 1.1669, 2.2205
  , 1.2771, 1.2647, 0.9709
  \multirow{3}{*}{6-1}, 0.8415, 0.8343, 0.8556
  , 1.1901, 1.1669, 1.9494
  , 1.3144, 1.3120, 0.1826
  \multirow{3}{*}{6-2}, 0.8378, 0.8366, 0.1432
  , 1.2114, 1.1815, 2.4682
  , 1.3189, 1.3120, 0.5232
  \multirow{3}{*}{6-4}, 0.8498, 0.8437, 0.7178
  , 1.2713, 1.2394, 2.5092
  , 1.3250, 1.3122, 0.9660
  \multirow{3}{*}{6-6}, 0.8519, 0.8513, 0.0704
  , 1.3147, 1.3100, 0.3575
  , 1.3474, 1.3356, 0.8758
}\comparison

\subsection*{Adquisici\'on y Procesamiento de Datos}

Se complet\'o el montaje del sistema de p\'endulos acoplados seg\'un lo
planificado. Se realizaron mediciones de la evoluci\'on angular de cada
p\'endulo utilizando el sensor Cassy.

Se recopilaron un total de 15 series de datos temporales. Estas series
cubren:
\begin{itemize}
  \item Cinco configuraciones experimentales distintas del sistema.
  \item Tres condiciones iniciales diferentes para cada configuraci\'on.
\end{itemize}

Los datos crudos ($\theta_i(t)$ para cada p\'endulo) fueron procesados
mediante un c\'odigo en Python. Los objetivos principales del procesamiento
fueron:
\begin{enumerate}
  \item Generar gr\'aficas de la evoluci\'on angular $\theta_i(t)$ para
    visualizar el comportamiento din\'amico.
  \item Determinar las frecuencias angulares principales de oscilaci\'on
    ($f_i$) para cada p\'endulo en cada serie, aplicando la
    Transformada R\'apida de Fourier (FFT) a los datos temporales.
\end{enumerate}

\subsection*{An\'alisis de Frecuencias Angulares Experimentales}

Las frecuencias angulares principales identificadas se resumen en el
\cref{tab:frequencies}.
\begin{table*}[htbp!]
  \centering
  \caption{Frecuencias angulares principales de oscilaci\'on ($f_i$)
    identificadas para cada p\'endulo, seg\'un la configuraci\'on
  experimental y las condiciones iniciales aplicadas.}
  \label{tab:frequencies}
  \pgfplotstabletypeset[
  every head row/.style={
    before row=\toprule,
    after row=\midrule
  },
  every last row/.style={after row=\bottomrule},
  columns/config/.style={
    string type,
    column name={Configuraci\'on},
  },
  columns/mode/.style={
    string type,
    column name={Condici\'on Inicial},
  },
  columns/freq1/.style={
    column name=$f_1 [\si{\Hz}]$,
    fixed,
    fixed zerofill,
    precision=3,
  },
  columns/freq2/.style={
    column name=$f_2 [\si{\Hz}]$,
    fixed,
    fixed zerofill,
    precision=3,
  },
  columns/freq3/.style={
    column name=$f_3 [\si{\Hz}]$,
    fixed,
    fixed zerofill,
    precision=3,
  },
  every nth row={3}{before row=\midrule},
  columns={config, mode, freq1, freq2, freq3}
  ]{\datafreq} % \datafreq debe apuntar al archivo de datos
\end{table*}

Del an\'alisis preliminar de los valores en \cref{tab:frequencies},
se observan los siguientes patrones:

\textbf{P\'endulo 2 (el m\'as largo):}
\begin{itemize}
  \item Frecuencia angular usual: consistentemente alrededor de
    \qty{0.844}{\Hz}.
  \item Desviaci\'on est\'andar: \num{0.009}. Esto indica una notable
    regularidad en su comportamiento oscilatorio a esta frecuencia.
\end{itemize}

\textbf{P\'endulo 1:}
\begin{itemize}
  \item Se identifican dos agrupaciones principales de frecuencias:
    \begin{itemize}
      \item En torno a \qty{1.311}{\Hz} (desv. est. \num{0.019}).
      \item Cercana a \qty{0.843}{\Hz} (desv. est. \num{0.008}).
    \end{itemize}
\end{itemize}

\textbf{P\'endulo 3:}
\begin{itemize}
  \item Comportamiento similar al p\'endulo 1, con valores ligeramente
    distintos:
    \begin{itemize}
      \item Frecuencias predominantes alrededor de \qty{1.255}{\Hz}
        (desv. est. \num{0.063}).
      \item Frecuencias alrededor de \qty{0.843}{\Hz} (desv. est. \num{0.008}).
    \end{itemize}
  \item Adicionalmente, se detect\'o una frecuencia excepcionalmente baja de
    \qty{0.0021}{\Hz}. Esta se present\'o de manera aislada,
    \'unicamente en la Configuraci\'on 5-1 bajo las condiciones
    iniciales (100).
\end{itemize}

\textbf{Interpretaci\'on inicial:} La aparici\'on de m\'ultiples frecuencias
dominantes para los p\'endulos laterales (1 y 3) sugiere la excitaci\'on
selectiva de diferentes modos normales del sistema. La manifestaci\'on de
estos modos parece depender de la configuraci\'on espec\'ifica de
acoplamiento y de las condiciones iniciales impuestas.

\textbf{An\'alisis detallado de \cref{tab:frequencies}:}
\begin{itemize}
  \item \textbf{Condici\'on inicial (010)} (excitaci\'on \'unica del p\'endulo
    central): Los tres p\'endulos tienden a oscilar con una frecuencia
    dominante com\'un o muy similar, independientemente de la
    configuraci\'on de acoplamiento. Esto podr\'ia indicar la
    excitaci\'on preferente de un modo normal en el que los tres
    p\'endulos participan de manera sincronizada.
  \item \textbf{Otras condiciones iniciales:} El p\'endulo 3 generalmente
    muestra una frecuencia principal inferior a la del p\'endulo 1.
    Una excepci\'on se observa en la configuraci\'on 6-6, donde la
    frecuencia principal del p\'endulo 3 supera a la del p\'endulo 1.
\end{itemize}

\textbf{Factores que influyen en las frecuencias observadas:}
\begin{itemize}
  \item La menor frecuencia caracter\'istica del p\'endulo 2 se atribuye
    principalmente a su mayor longitud y masa (mayor momento de inercia)
    en comparaci\'on con los p\'endulos laterales.
  \item Las diferencias en las frecuencias principales entre los p\'endulos 1 y
    3, y su variaci\'on entre configuraciones, son consecuencia de c\'omo
    los diferentes puntos de acople de los resortes modifican la
    interacci\'on. La altura de fijaci\'on de los resortes influye en los
    torques de acoplamiento y, por ende, en la 'rigidez efectiva
    angular' del acoplamiento. Esto afecta las caracter\'isticas de los
    modos normales.
  \item \textbf{Configuraci\'on 6-6:} Sus puntos de acople espec\'ificos
    parecen facilitar una interacci\'on que favorece un modo donde el
    p\'endulo 3 oscila a una frecuencia mayor que el 1.
  \item \textbf{Frecuencia an\'omala de \qty{0.0021}{\Hz} (P3, Config. 5-1,
    CI (100)):} En esta configuraci\'on, un punto de anclaje del
    resorte est\'a muy pr\'oximo al pivote. Esto resulta en una 'rigidez
    efectiva angular' muy d\'ebil para ciertos patrones de movimiento,
    llevando a frecuencias de oscilaci\'on extremadamente bajas para el
    modo asociado.
  \item \textbf{Configuraci\'on 6-4, CI (100):} Los p\'endulos laterales (1 y 3)
    presentan la misma frecuencia principal. Esto podr\'ia indicar la
    excitaci\'on de un modo normal con alta simetr\'ia en el movimiento
    de los p\'endulos externos.
\end{itemize}

\subsection*{C\'alculo de Frecuencias del Modelo Te\'orico}

Se resolvi\'o el problema de valores propios para el sistema (representado
en forma matricial para obtener las frecuencias te\'oricas de los modos normales
para las cinco configuraciones experimentales.

Los resultados te\'oricos se presentan en el \cref{tab:theoric-freq}.
Se identifican tres frecuencias de modo normal distintas para cada
configuraci\'on, como es de esperar para un sistema con tres grados de
libertad.
\begin{table*}[htbp!]
  \centering
  \caption{Frecuencias te\'oricas de los modos normales ($f_{0i}$)
    calculadas para el sistema, seg\'un la configuraci\'on de
  acoplamiento.}
  \label{tab:theoric-freq}
  \pgfplotstabletypeset[
  every head row/.style={
    before row=\toprule,
    after row=\midrule
  },
  every last row/.style={after row=\bottomrule},
  columns/config/.style={
    string type,
    column name={Configuraci\'on},
  },
  columns/f2/.style={ % Revisar si los nombres f1,f2,f3 son consistentes
    column name=$f_{01} [\si{\Hz}]$, % f01, f02, f03 es más estándar
    fixed,
    fixed zerofill,
    precision=3,
  },
  columns/f1/.style={
    column name=$f_{02} [\si{\Hz}]$,
    fixed,
    fixed zerofill,
    precision=3,
  },
  columns/f3/.style={
    column name=$f_{03} [\si{\Hz}]$,
    fixed,
    fixed zerofill,
    precision=3,
  },
  columns={config, f2, f1, f3} % Asegurar orden correcto de columnas
  ]{\theoric} % \theoric debe apuntar al archivo de datos
\end{table*}

\textbf{Observaciones sobre las frecuencias te\'oricas (\cref{tab:theoric-freq}):}
\begin{itemize}
  \item \textbf{Tendencia general:} Una mayor distancia entre el punto de
    acople del resorte y el pivote del p\'endulo tiende a correlacionarse
    con un aumento en las frecuencias de los modos normales.
  \item \textbf{Valores espec\'ificos:} Se observa con regularidad una
    frecuencia te\'orica cercana a \qty{0.8}{\Hz}. Otras est\'an
    agrupadas en torno a \qty{1.3}{\Hz} y, en algunos casos, pr\'oximas a
    \qty{1.6}{\Hz}.
\end{itemize}

\textbf{Comparaci\'on inicial con datos experimentales:}
\begin{itemize}
  \item Las frecuencias te\'oricas obtenidas, en general, no difieren
    significativamente de los valores experimentales predominantes
    (ver \cref{tab:frequencies}).
  \item \textbf{Excepci\'on notable:} La frecuencia experimental an\'omalamente
    baja de \qty{0.0021}{\Hz} (Config. 5-1, CI (100), P3) no tiene una
    contraparte directa en el espectro te\'orico calculado.
  \item \textbf{Concordancia general:} A pesar de la excepci\'on, la
    concordancia para las dem\'as frecuencias principales sugiere que el
    modelo te\'orico linealizado (basado en peque\~nas oscilaciones)
    aproxima razonablemente el comportamiento din\'amico del sistema.
\end{itemize}

\subsection*{Comparaci\'on Detallada: Frecuencias Experimentales vs. Te\'oricas}

En la \cref{tab:comparison} se presenta una comparaci\'on directa entre las
frecuencias experimentales ($f_\text{ex}$) predominantes (para modos
discernibles) y sus correspondientes frecuencias te\'oricas ($f_\text{te}$).
Se incluye el error relativo porcentual:
$E = \mid \left(1 - \frac{f_\text{te}}{f_\text{ex}}\right) \mid 100\%$.
\begin{table*}[htbp!]
  \centering
  \caption{Comparaci\'on entre las frecuencias experimentales
    predominantes ($f_\text{ex}$) y las frecuencias te\'oricas
    ($f_\text{te}$) para los modos normales, junto con el error
  relativo porcentual.}
  \label{tab:comparison}
  \pgfplotstabletypeset[
  every head row/.style={
    before row=\toprule,
    after row=\midrule
  },
  every last row/.style={after row=\bottomrule},
  columns/config/.style={
    string type,
    column name={Configuraci\'on},
  },
  columns/fe/.style={
    column name=$f_\text{ex} [\si{\Hz}]$,
    fixed,
    fixed zerofill,
    precision=3,
  },
  columns/ft/.style={
    column name=$f_\text{te} [\si{\Hz}]$,
    fixed,
    fixed zerofill,
    precision=3,
  },
  columns/err/.style={
    column name={Error Relativo $[\%]$},
    fixed,
    fixed zerofill,
    precision=3,
  },
  every nth row={3}{before row=\midrule},
  columns={config, fe, ft, err}
  ]{\comparison} % \comparison debe apuntar al archivo de datos
\end{table*}

\textbf{An\'alisis de errores relativos (\cref{tab:comparison}):}
\begin{itemize}
  \item Los errores relativos porcentuales var\'ian considerablemente.
  \item Error m\'as grande observado (sin considerar la frecuencia an\'omala
    de \qty{0.0021}{\Hz}, no incluida aqu\'i): \qty{2.509}{\percent}.
  \item Error m\'as bajo registrado: \qty{0.036}{\percent}.
  \item Estos valores indican que, si bien existen desviaciones, el modelo
    te\'orico predice las frecuencias de los modos normales con un grado
    de precisi\'on generalmente alto para la mayor\'ia de los casos.
\end{itemize}

\subsection*{Visualizaci\'on: Casos Representativos de Oscilaci\'on y Espectros}

Se seleccionaron seis casos del conjunto de datos experimentales como los
m\'as representativos de los diversos comportamientos din\'amicos observados.
Para cada caso, se generaron figuras que muestran:
(a) Evoluci\'on temporal del desplazamiento angular $\theta_i(t)$.
(b) Espectro de amplitudes (FFT) correspondiente.

Estas representaciones gr\'aficas se detallan en las
\cref{fig:111-11,fig:010-51,fig:101-61,fig:010-62,fig:100-66,fig:010-66}.

\begin{figure}[htbp!]
  \centering
  \includegraphics[width=0.8\linewidth]{./Figures/111_11_filtrado.pdf}
  \caption{Evoluci\'on temporal $\theta_i(t)$ y espectro FFT.
  Configuraci\'on 1-1, CI (111).}
  \label{fig:111-11}
\end{figure}

\begin{figure}[htbp!]
  \centering
  \includegraphics[width=0.8\linewidth]{./Figures/010_15_filtrado.pdf}
  \caption{Evoluci\'on temporal $\theta_i(t)$ y espectro FFT.
  Configuraci\'on 5-1, CI (010).}
  \label{fig:010-51}
\end{figure}

\begin{figure}[htbp!]
    \centering
    \includegraphics[width=0.8\linewidth]{./Figures/101_16_filtrado.pdf}
    \caption{Evoluci\'on temporal $\theta_i(t)$ y espectro FFT.
        Configuraci\'on 6-1, CI (101).}
    \label{fig:101-61}
\end{figure}

\begin{figure}[htbp!]
    \centering
    \includegraphics[width=0.8\linewidth]{./Figures/010_26_filtrado.pdf}
    \caption{Evoluci\'on temporal $\theta_i(t)$ y espectro FFT.
        Configuraci\'on 6-2, CI (010).}
    \label{fig:010-62}
\end{figure}

\begin{figure}[htbp!]
    \centering
    \includegraphics[width=0.8\linewidth]{./Figures/001_66_filtrado.pdf}
    \caption{Evoluci\'on temporal $\theta_i(t)$ y espectro FFT.
        Configuraci\'on 6-6, CI (100).}
    \label{fig:100-66}
\end{figure}

\begin{figure}[htbp!]
    \centering
    \includegraphics[width=0.8\linewidth]{./Figures/010_66_filtrado.pdf}
    \caption{Evoluci\'on temporal $\theta_i(t)$ y espectro FFT.
        Configuraci\'on 6-6, CI (010).}
    \label{fig:010-66}
\end{figure}

\textbf{An\'alisis de casos espec\'ificos:}

\textbf{\Cref{fig:111-11} (Config. 1-1, CI (111)):}
\begin{itemize}
  \item Cada p\'endulo exhibe un pico de frecuencia principal distintivo en su
    FFT; no comparten una \'unica frecuencia dominante com\'un.
  \item Picos espectrales relativamente anchos: sugiere duraci\'on limitada
    de coherencia o amortiguamiento significativo.
  \item Amortiguamiento esperado: debido a interacciones (acoplamiento,
    fricci\'on) y complejidad del movimiento (laterales desfasados
    respecto al central).
  \item P\'endulo 2: comportamiento temporal con modulaciones en amplitud,
    no arm\'onico simple.
\end{itemize}

\textbf{\Cref{fig:010-51} (Config. 5-1, CI (010)):}
\begin{itemize}
  \item Picos espectrales m\'as definidos y estrechos: comportamiento m\'as
    regular, menos amortiguado para frecuencias dominantes.
  \item Los tres p\'endulos comparten la misma frecuencia principal (reafirma
    tendencia para CI (010)).
  \item Componentes espectrales secundarias comunes, con diferentes
    amplitudes relativas entre p\'endulos.
  \item P\'endulo 3: menor amplitud en componentes espectrales (posiblemente
    menor eficiencia en transmisi\'on de energ\'ia por geometr\'ia de
    acople).
  \item Evoluci\'on temporal m\'as estable y coherente; disminuci\'on gradual
    de amplitudes (amortiguamiento residual).
\end{itemize}

\textbf{\Cref{fig:101-61} (Config. 6-1, CI (101)):}
\begin{itemize}
  \item An\'alisis espectral revela tres picos de frecuencia principales
    distintos (sugiere excitaci\'on de m\'ultiples modos normales).
  \item P\'endulo 3: mayor amplitud en su frecuencia principal y pico
    espectral m\'as estrecho comparado con P1 (relacionado con
    naturaleza del acoplamiento y excitaci\'on inicial de P3).
  \item P\'endulo 2: oscilaciones de amplitud muy reducida (consistente con
    modo normal donde laterales se mueven desfasados y P2 tiene
    participaci\'on m\'inima).
  \item Evoluci\'on temporal: intercambio de energ\'ia entre P1 y P3
    (pulsaciones/batidos), esperado si frecuencias de modos son cercanas
    y hay transferencia de energ\'ia.
\end{itemize}

\textbf{\Cref{fig:010-62} (Config. 6-2, CI (010)):}
\begin{itemize}
  \item P\'endulo 2 (espectro): frecuencia principal con amplitud prominente y
    pico estrecho. Componente secundaria a aprox. $4 f_P$.
    Comportamiento temporal similar a M.A.S.
  \item P\'endulo 1 (espectro): comparte $f_P$ de P2. Otro pico significativo
    a frecuencia $> 3 f_P$. Movimiento no estrictamente M.A.S.,
    pero con clara periodicidad.
  \item P\'endulo 3 (espectro): dos picos con amplitudes comparables,
    frecuencias corresponden aprox. con valores te\'oricos esperados
    (\qty{0.83}{\Hz} y \qty{1.2}{\Hz}). Amplitudes bajas, oscilaci\'on
    de escasa magnitud.
\end{itemize}

\textbf{\Cref{fig:100-66} (Config. 6-6, CI (100)):}
\begin{itemize}
  \item Geometr\'ia de acople facilita transmisi\'on efectiva de energ\'ia.
  \item An\'alisis espectral: los tres p\'endulos comparten componentes de
    frecuencia en mismas ubicaciones espectrales.
  \item P1 y P3 comparten frecuencia principal com\'un (m\'as alta).
  \item P2 tiene su propia $f_P$ (menor), pero exhibe picos secundarios en
    frecuencias donde P1 y P3 oscilan predominantemente (participa en
    modos de mayor frecuencia).
  \item Comportamiento (todos oscilan significativamente) distingue esta
    combinaci\'on, resaltando papel determinante de posici\'on de acoples.
\end{itemize}
