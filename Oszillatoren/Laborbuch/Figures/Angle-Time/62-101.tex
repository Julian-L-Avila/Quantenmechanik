\documentclass[tikz]{standalone}

\usepackage{pgfplots}
\usepackage{pgfplotstable}
\pgfplotsset{compat=1.18}
\usepackage{tikz}
\usepackage[dvipsnames]{xcolor}
\usepackage{siunitx}

\sisetup{separate-uncertainty}

\begin{document}

\begin{tikzpicture}
	\begin{axis}[
		width = 30cm,
		height = 11cm,
		title = {Configuración 6-1, Condición inicial 101},
		xlabel = {\(t \, [\unit{\second}]\)},
		ylabel = {\(\theta [\unit{\degree}]\)},
		xmin = 0.0, xmax = 12.0,
		ymin = -25.0, ymax = 25.0,
		xtick distance=2, minor x tick num=1,
		ytick distance=10, minor y tick num=1,
		grid = major,
		major grid style = {lightgray!60, thin},
		minor grid style = {lightgray!30, very thin},
		axis line style = {thick},
		tick style={thick, black},
		legend pos = north east,
		legend style = {
			draw=none,
			fill=none,
		},
		title style={font=\bfseries\Large},
		label style={font=\bfseries\large},
		tick label style={font=\normalsize},
		]

		\addplot[
			ultra thick,
			color = Orchid,
			smooth,
			] table[
			x expr={\thisrowno{0} - 3.0},
			y expr={\thisrowno{1}},
			]{../../../Data/101_16.tsv};
		\addlegendentry{$\theta_1$}

		\addplot[
			ultra thick,
			color = ForestGreen,
			] table[
			x expr={\thisrowno{0} - 3.0},
			y expr={\thisrowno{2}},
			]{../../../Data/101_16.tsv};
		\addlegendentry{$\theta_2$}

		\addplot[
			ultra thick,
			color= Cyan,
			smooth,
			] table[
			x expr={\thisrowno{0} - 3.0},
			y expr={\thisrowno{3}},
			]{../../../Data/101_16.tsv};
		\addlegendentry{$\theta_3$}
	\end{axis}
\end{tikzpicture}

\end{document}
