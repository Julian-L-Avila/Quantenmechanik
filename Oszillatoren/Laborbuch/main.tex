\documentclass[fontsize=11pt,
  paper=a4paper,
  twoside,
  captions=tableheading,
  index=totoc,
  hyperref]{labbook}

\usepackage[bottom=10em]{geometry}
\usepackage[spanish]{babel}
\usepackage[utf8]{inputenc}
\usepackage[T1]{fontenc}
\usepackage[osf]{mathpazo}
\linespread{1.05}\selectfont
\usepackage[scaled=.88]{beramono}
\usepackage[scaled=.86]{berasans}
\usepackage{booktabs,array}
\usepackage{biblatex}
\usepackage{amsmath}
\usepackage{graphicx}
\usepackage{etoolbox}
\usepackage[norule]{footmisc}
\usepackage{lastpage}
\usepackage[dvipsnames]{xcolor}
\definecolor{titleblue}{rgb}{0.16,0.24,0.64}
\definecolor{linkcolor}{rgb}{0,0,0.42}
\addtokomafont{title}{\Huge\color{titleblue}}
\addtokomafont{chapter}{\color{OliveGreen}}
\addtokomafont{section}{\color{Sepia}}
\addtokomafont{pagehead}{\normalfont\sffamily\color{gray}}
\addtokomafont{caption}{\footnotesize\itshape}
\addtokomafont{captionlabel}{\upshape\bfseries}
\addtokomafont{descriptionlabel}{\rmfamily}
\setcapwidth[r]{10cm}
\setkomafont{footnote}{\sffamily}
\deffootnote[4cm]{4cm}{1em}{\textsuperscript{\thefootnotemark}}
\DeclareFixedFont{\textaut}{T1}{phv}{bx}{n}{0.8cm}
\usepackage[nouppercase,headsepline]{scrlayer-scrpage}
\pagestyle{scrheadings}
\clearscrheadfoot

\automark[chapter]{chapter}
\ohead{\headmark}

\setlength{\headheight}{25pt}
\setheadsepline{.4pt}
\addtokomafont{headsepline}{\color{lightgray}}

\ofoot[\normalfont\normalcolor{\thepage\ |\  \pageref{LastPage}}]{\normalfont\normalcolor{\thepage\ |\  \pageref{LastPage}}}
\makeatletter
\patchcmd{\addchap}{\if@openright\cleardoublepage\else\clearpage\fi}{\par}{}{}
\makeatother
\renewcommand*{\chapterpagestyle}{scrheadings}

\usepackage{chngcntr}
\counterwithout{figure}{labday}
\counterwithout{equation}{labday}

\usepackage[
  pdfauthor={Rodriguez S, Torres L, Avila J},
  pdftitle={Laboratory Journal},
  pdfsubject={Couple Oscillators},
  bookmarksopen=true,
  linktocpage=true,
  urlcolor=linkcolor,
  citecolor=linkcolor,
  linkcolor=linkcolor,
  pdfpagelabels=true,
  plainpages=false,
  colorlinks=true,
  bookmarks=true,
  pdfview=FitB]{hyperref}

\usepackage[stretch=10]{microtype}

%----------------------------------------------------------------------------------------
%	DEFINITION OF EXPERIMENTS
%----------------------------------------------------------------------------------------

% Template: \newexperiment{<abbrev>}[<short form>]{<long form>}
% <abbrev> is the reference to use later in the .tex file in \experiment{}, the <short form> is only used in the table of contents and running title - it is optional, <long form> is what is printed in the lab book itself

\newexperiment{example}[Example experiment]{This is an example experiment}
\newexperiment{example2}[Example experiment 2]{This is another example experiment}
\newexperiment{example3}[Example experiment 3]{This is yet another example experiment}

\newsubexperiment{subexp_example}[Example sub-experiment]{This is an example sub-experiment}
\newsubexperiment{subexp_example2}[Example sub-experiment 2]{This is another example sub-experiment}
\newsubexperiment{subexp_example3}[Example sub-experiment 3]{This is yet another example sub-experiment}

%----------------------------------------------------------------------------------------

\begin{document}

%----------------------------------------------------------------------------------------
%	TITLE PAGE
%----------------------------------------------------------------------------------------

\title{\fontsize{40pt}{40pt}\selectfont{Osciladores Acoplados \\[1cm]
\textaut{Bitácora de laboratorio}}}

\author{
  \textaut{Sebastian Rodriguez} \and \textaut{Laura Torres}
  \and \textaut{Julian Avila}
}
\date{}

\maketitle

\printindex
\tableofcontents
\newpage

\begin{addmargin}[4cm]{0cm}

\pagestyle{scrheadings}

%----------------------------------------------------------------------------------------
%	LAB BOOK CONTENTS
%----------------------------------------------------------------------------------------

\labday{Miércoles 23, Abril 2025}


\labday{Miércoles 24, Abril 2025}

El día de hoy se realizo el desarrollo teorico del problema de los tres pendulos físicos, acoplados por los resortes, donde el sistema es el siguiente:
\begin{figure}[h!]
 
  \includegraphics[width=0.8\textwidth]{Figures/IM1.jpeg}

  \caption{Sistema de tres péndulos físicos acoplados por resortes.}
  \label{fig:sistema_pendulos}
\end{figure}
Donde el resultado de la sumatoria de torques para cada pendulo genera el siguiente sistema de ecuaciones:

\begin{equation}
\begin{aligned}
  \ddot{\theta}_1 =\; & \theta_1 \left( \frac{(cl)^2 - x_{cm1} m_1 g}{I_1} \right) + \theta_2 \left( -\frac{k_1 (cl)^2}{I_1} \right) \\
  \ddot{\theta}_2 =\; & \theta_1 \left( \frac{k_1 (cl)^2}{I_2} \right) + \theta_2 \left( -\frac{k_1 (cl)^2}{I_2} + \frac{k_2 (dl)^2}{I_2} + \frac{x_{cm2} m_2 g}{I_2} \right) + \theta_3 \left( \frac{k_2 (dl)^2}{I_2} \right) \\
  \ddot{\theta}_3 =\; & \theta_2 \left( \frac{k_2 (dl)^2}{I_3} \right) + \theta_3 \left( -\frac{k_2 (dl)^2}{I_3} - \frac{x_{cm3} m_3 g}{I_3} \right)
  \end{aligned}
\end{equation}

\experiment{Problema inicial}

Se plantea realizar un montaje experimental de tres péndulos rígidos acoplados,
con el objetivo de que bajo pequeñas oscilaciones, se pueda modelar el sistema
como también visualizar los modos normales del mismo y su movimiento bajo los
mismos.

La \cref{fig:system-diagram} muestra la idea inicial del sistema,
donde la longitud del péndulo central (\( \ell \)) es el doble de las otras dos,
están separadas peor una distancia \( a \), cuentan con 5 posibles puntos de
acople a lo largo del cuerpo rígido.

\begin{figure}
	\centering
	\includegraphics[width=0.7\textwidth]{./Figures/}
\caption{}
\label{fig:}
\end{figure}


%----------------------------------------------------------------------------------------

\end{addmargin}

%----------------------------------------------------------------------------------------
%	BIBLIOGRAPHY
%----------------------------------------------------------------------------------------

%----------------------------------------------------------------------------------------

\end{document}
